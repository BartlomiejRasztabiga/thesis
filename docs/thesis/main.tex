%-----------------------------------------------
%  Engineer's & Master's Thesis Template
%  Copyleft by Artur M. Brodzki & Piotr Woźniak
%  Warsaw University of Technology, 2019-2022
%-----------------------------------------------

\documentclass[
    bindingoffset=5mm,  % Binding offset
    footnoteindent=3mm, % Footnote indent
    hyphenation=true    % Hyphenation turn on/off
]{src/wut-thesis}

\graphicspath{{tex/img/}} % Katalog z obrazkami.
\addbibresource{bibliografia.bib} % Plik .bib z bibliografią

%-------------------------------------------------------------
% Wybór wydziału:
%  \facultyeiti: Wydział Elektroniki i Technik Informacyjnych
%  \facultymeil: Wydział Mechaniczny Energetyki i Lotnictwa
% --
% Rodzaj pracy: \EngineerThesis, \MasterThesis
% --
% Wybór języka: \langpol, \langeng
%-------------------------------------------------------------
\facultyeiti    % Wydział Elektroniki i Technik Informacyjnych
\EngineerThesis % Praca inżynierska
\langpol % Praca w języku polskim

\begin{document}

%------------------
% Strona tytułowa
%------------------
% TODO na pewno? a nie instytut promotora?
\instytut{Automatyki i Informatyki Stosowanej}
\kierunek{Informatyka}
\specjalnosc{Inżynieria Oprogramowania}
\title{
    Zastosowanie architektur mikroserwisowych\\
    w tworzeniu skalowalnych aplikacji webowych\\ 
    na przykładzie aplikacji do obsługi zamówień restauracyjnych
}
% Title in English for English theses
% In English theses, you may remove this command
\engtitle{
    Unnecessarily long and complicated thesis' title \\
    difficult to read, understand and pronounce
}
% Title in Polish for English theses
% Use it only in English theses
\poltitle{
    Niepotrzebnie długi i skomplikowany tytuł pracy \\
    trudny do przeczytania, zrozumienia i wymówienia
}
\author{Bartłomiej Patryk Rasztabiga}
\album{304117}
\promotor{mgr inż. Zbigniew Szymański}
\date{\the\year}
\maketitle

%-------------------------------------
% Streszczenie po polsku dla \langpol
% English abstract if \langeng is set
%-------------------------------------
\cleardoublepage % Zaczynamy od nieparzystej strony
\abstract \lipsum[1-3]
\keywords XXX, XXX, XXX

%----------------------------------------
% Streszczenie po angielsku dla \langpol
% Polish abstract if \langeng is set
%----------------------------------------
\clearpage
\secondabstract \kant[1-3]
\secondkeywords XXX, XXX, XXX

\pagestyle{plain}

%--------------
% Spis treści
%--------------
\cleardoublepage % Zaczynamy od nieparzystej strony
\tableofcontents

%------------
% Rozdziały
%------------
\cleardoublepage % Zaczynamy od nieparzystej strony
\pagestyle{headings}

\clearpage % Rozdziały zaczynamy od nowej strony.
\section{Wstęp}

\subsection{Motywacja}

Wybór tematu pracy dyplomowej został podyktowany chęcią pogłębienia wiedzy oraz zaprezentowania technik tworzenia skalowalnych systemów mikroserwisowych na podstawie konkretnej aplikacji.

Wybraną przez Autora dziedziną do tego celu jest branża gastronomiczna. W ostatnich latach, wraz z rozwojem techniki, zaczęło powstawać wiele aplikacji służących do zamawiania jedzenia z restauracji. Aplikacje te cieszą się ogromną popularnością, więc ich twórcy muszą zapewnić odpowiednią skalowalność systemu, aby móc obsłużyć duży ruch. Jednym z rozwiązań tego problemu jest zastosowanie skalowalnych architektur, np. architektury mikroserwisowej.

Jednym z największych wyzwań stojących przed twórcami takich systemów jest zapewnienie niezawodności i ciągłości działania, co jest kluczowe w branży, gdzie każda minuta przestoju może skutkować znaczącymi stratami finansowymi i reputacyjnymi. Aplikacje służące do zamawiania jedzenia muszą być nie tylko funkcjonalne i intuicyjne dla użytkownika, ale także wydajne i elastyczne pod względem technologicznym.

Architektura mikroserwisowa jest jednym z najbardziej innowacyjnych podejść do tworzenia aplikacji webowych. W ostatnich latach zyskała ona na popularności, a jej zalety są doceniane przez coraz większą liczbę programistów. Wymaga ona jednak zupełnie innego podejścia niż architektura monolityczna, która jest najczęściej stosowana w aplikacjach tego typu. Różnice wynikają np. z innego podejścia do transakcyjności, synchronizacji danych oraz komunikacji między komponentami.

Zastosowanie architektury mikroserwisowej pozwala na elastyczne zarządzanie złożonymi systemami, umożliwiając jednocześnie łatwiejszą integrację z różnorodnymi usługami zewnętrznymi, takimi jak systemy płatności, zarządzanie dostawami czy integracja z mediami społecznościowymi. To podejście zapewnia również łatwiejsze skalowanie poszczególnych elementów systemu, co jest niezbędne w branży, gdzie wzrost popularności restauracji może gwałtownie zwiększyć zapotrzebowanie na zasoby informatyczne.

W pracy dyplomowej Autor prezentuje przykład aplikacji webowej zbudowanej zgodnie z najlepszymi praktykami architektury mikroserwisowej. Aplikacja ta będzie służyła do obsługi zamówień restauracyjnych z trzech perspektyw: klienta, restauracji oraz dostawcy.

Praca ta ma na celu nie tylko zaspokojenie potrzeb akademickich, ale także praktyczne pokazanie, jak nowoczesne technologie informatyczne mogą być wykorzystywane do rozwiązywania realnych problemów biznesowych. Opracowany system ma stanowić modelowy przykład aplikacji, która dzięki zastosowaniu architektury mikroserwisowej, może być łatwo dostosowywana do zmieniających się potrzeb rynku i rosnącej liczby użytkowników, co ma szczególne znaczenie w szybko rozwijającej się branży gastronomicznej.

\subsection{Cel pracy}

Celem niniejszej pracy było zaprojektowanie, stworzenie, wdrożenie oraz przetestowanie systemu informatycznego do obsługi zamówień restauracyjnych, opartego na architekturze mikroserwisowej. System ten ma na celu zapewnienie wydajnego, elastycznego i skalowalnego rozwiązania, które będzie mogło sprostać wymaganiom i oczekiwaniom trzech głównych grup użytkowników: zamawiających, restauracji oraz kurierów.

\subsection{Charakterystyka problemu}

Zastosowana architektura wymagała odpowiedniego podziału systemu informatycznego na mikroserwisy, które komunikują się między sobą w celu spełnienia wymagań biznesowych. System został podzielony na dwie części: kliencką i serwerową. Część kliencka jest aplikacją webową, uruchamianą w przeglądarce klienta, komunikującą się z częścią serwerową przy pomocy interfejsu REST \cite{rest}. Część serwerowa została wykonana we wcześniej wspomnianej architekturze mikroserwisowej i zawiera całą logikę biznesową systemu. W pracy omówiono zastosowane techniki przy tworzeniu modułów oraz korzyści wynikające z wykorzystania tej specyficznej architektury. Ponadto, zaprezentowano krótkie zestawienie tej architektury z monolityczną, analizując ich dopasowanie do omawianego problemu.

\subsection{Przegląd podobnych rozwiązań}

W ramach tej pracy inżynierskiej Autor nie był w stanie przeprowadzić dogłębnego porównania projektowanego systemu z istniejącymi na rynku rozwiązaniami takimi jak Uber Eats, Wolt, Pyszne.pl, Bolt Food \cite{fooddeliveryapps}. Powodem tego jest ograniczony dostęp do szczegółowych informacji na temat ich architektury serwerowej i rozwiązań technologicznych. Ponadto, porównanie funkcjonalności z perspektywy użytkownika końcowego nie było głównym celem pracy, gdyż nacisk położono nie na rozwój licznych funkcjonalności, lecz na badanie aspektów technicznych związanych z tworzeniem skalowalnych aplikacji webowych. W związku z tym analiza skupia się bardziej na kwestiach architektonicznych i wydajnościowych, a nie na bezpośredniej konkurencji z funkcjami oferowanymi przez wymienione platformy.


\subsection{Struktura pracy}

Kolejne rozdziały prezentują kolejne etapy prac nad prezentowanym systemem.

Na początku zostaną omówione popularne techniki tworzenia skalowalnych aplikacji webowych wykorzystane w niniejszej pracy inżynierskiej, w tym architektura mikroserwisowa. Zostanie następnie przedstawione jej porównanie ze standardową architekturą monolityczną.

Kolejny rozdział będzie poświęcony analizie wymagań budowanego systemu. Przedstawiony zostanie słownik dziedziny problemu, metoda wydzielania mikroserwisów, wymagania funkcjonalne i niefunkcjonalne.

W kolejnym rozdziale zostanie przedstawiona proponowana architektura systemowa i wdrożeniowa systemu, wraz z jej modelem C4. Opisane zostaną zastosowane mechanizmy persystencji danych, przechowywania i transportu komunikatów. Pokrótce opisane zostaną również mechanizmy uwierzytelniania i autoryzacji oraz zastosowane integracje z zewnętrznymi usługami.

Następny rozdział skupi się na implementacji systemu. Będzie zawierać szczegółowy opis procesu tworzenia obu segmentów systemu informatycznego, z uwzględnieniem wykorzystanych narzędzi, technik, ich zastosowania oraz przykładów ich użycia w systemie.

Przedostatni rozdział będzie poświęcony testowaniu systemu. Opisane zostaną różne rodzaje testów i sposób ich wykorzystania do sprawdzenia zgodności systemu z wymaganiami.

W końcowym rozdziale zostaną przedstawione wnioski wynikające z implementacji systemu i przeprowadzonych testów wydajnościowych. Na koniec przedstawione zostaną także perspektywy dalszego rozwoju systemu.         % Wygodnie jest trzymać każdy rozdział w osobnym pliku.
\clearpage % Rozdziały zaczynamy od nowej strony.
\section{Techniki tworzenia skalowalnych aplikacji webowych}

Taki rozdział w ogóle będzie przydatny?

\subsection{Architektura mikroserwisowa}

XXX

\subsection{DDD}

DDD, CQRS, Event Sourcing, Sagi indent

To są wzorce projektowe czy architektoniczne?

\subsection{CQRS}

XXX

\subsection{Event Sourcing}

XXX

\subsection{Wzorzec Saga}

XXX    % Umożliwia to również łatwą migrację do nowej wersji szablonu:
\clearpage % Rozdziały zaczynamy od nowej strony.

\section{Analiza wymagań systemu}

Przed rozpoczęciem implementacji systemu należy go przeanalizować pod kątem wymagań funkcjonalnych oraz niefunkcjonalnych. Twórca powinien uprzednio poznać występujące rodzaje danych, udostępniane na nich operacje oraz klasy użytkowników.

W systemach o architekturze mikroserwisowej należy również odpowiednio wydzielić serwisy, stanowiące spójną całość.

W tym rozdziale zostaną też przedstawione najważniejsze terminy i pojęcia używane w aplikacji, jej wymagania oraz przypadki użycia.

\subsection{Słownik dziedziny problemu}

Słownik najważniejszych pojęć występujących w dziedzinie systemu. W nawiasach podano ich angielskie odpowiedniki, ponieważ w takiej formie zostały one użyte w kodzie aplikacji.

\begin{itemize}

    \item \textbf{Użytkownik zamawiający} (ang. \textit{Ordering User}) - encja reprezentująca użytkownika systemu, odpowiedzialnego za składanie zamówień. Użytkownik taki może definiować swoje adresy dostawy, oraz posiadać historię zamówień. Jest on definiowany przez nazwę oraz adres e-mail.
    \item \textbf{Restauracja} (ang. \textit{Restaurant}) - encja reprezentująca restaurację, która może być obsługiwana przez system. Restauracja posiada menu, które może być modyfikowane przez managera restauracji. Restauracja może być otwarta lub zamknięta, a jej manager może przyjmować lub odrzucać zamówienia. Restauracja jest definiowana przez nazwę, adres, menu oraz managera.
    \item \textbf{Kurier} (ang. \textit{Courier}) - encja reprezentująca dostawcę, który może być przypisany do zamówienia. Kurier może być dostępny lub niedostępny, a jego status może być aktualizowany samodzielne. Posiada on również lokalizację w postaci współrzędnych geograficznych. Kurier jest definiowany przez nazwę, adres e-mail oraz status.
    \item \textbf{Dostawa} (ang. \textit{Order Delivery}) - encja reprezentująca dostawę zamówienia przez kuriera z restauracji do użytkownika. Dostawa posiada status, który może być aktualizowany przez system. Dostawa jest definiowana przez zamówienie, kuriera, lokalizację restauracji, lokalizację użytkownika, obliczoną opłatę dla kuriera oraz status.
    \item \textbf{Zamówienie} (ang. \textit{Order}) - encja reprezentująca zamówienie użytkownika. Zamówienie posiada status, który może być aktualizowany przez system. Zamówienie jest definiowane przez użytkownika, restaurację, adres dostawy, listę dań oraz status.
    \item \textbf{Faktura} (ang. \textit{Invoice}) - encja reprezentująca fakturę wystawioną przez system. Faktura może być wystawiona dla użytkownika, restauracji lub kuriera. Faktura jest definiowana przez składniki, kwotę, datę wystawienia oraz adres e-mail odbiorcy.
    \item \textbf{Płatność} (ang. \textit{Order Payment}) - encja reprezentująca płatność za zamówienie. Płatność może być dokonana przez użytkownika. Płatność jest definiowana przez zamówienie, metodę płatności, kwotę oraz status.
    \item \textbf{Odbiora płatności} (ang. \textit{Payee}) - encja reprezentująca odbiorcę płatności. Odbiorca płatności może być restauracją lub kurierem. Odbiorca płatności jest definiowany przez nazwę, adres e-mail oraz balans kwotowy.
    \item \textbf{Płatnik} (ang. \textit{Payer}) - encja reprezentująca płatnika. Płatnik może być użytkownikiem lub systemem. Płatnik jest definiowany przez nazwę oraz adres e-mail.
    \item \textbf{Zamówienie restauracyjne} (ang. \textit{Restaurant Order}) - encja reprezentująca zamówienie złożone w restauracji. Zamówienie restauracyjne posiada status, który może być aktualizowany przez system. Zamówienie restauracyjne jest definiowane przez restaurację, listę dań oraz status.

\end{itemize}

\subsection{Burza Zdarzeń}

Aby móc podzielić system informatyczny na komponenty, później przekształcone w mikroserwisy, można zastosować np. technikę Burzy Zdarzeń (ang. Event Storming).

Burza Zdarzeń to metoda warsztatowa używana w projektowaniu i analizie systemów opartych na mikroserwisach. Została opracowana przez Alberto Brandoliniego i polega na interaktywnym modelowaniu procesów biznesowych poprzez identyfikację i dyskusję na temat "zdarzeń" mających znaczenie biznesowe. Uczestnicy, wykorzystując kolorowe karteczki, reprezentują różne aspekty systemu, takie jak wydarzenia, komendy, czy modele danych. Ta metoda sprzyja współpracy między różnymi zespołami, pomagając im lepiej zrozumieć procesy biznesowe i technologiczne oraz identyfikować potencjalne problemy i możliwości dla architektury mikroserwisowej.

\begin{figure}[!h]
    \centering \includegraphics[width=0.8\linewidth]{event_storming.png}
    \caption{Przykładowy rezultat sesji Burzy Zdarzeń \cite{event_storming_rys}}
\end{figure}


Rezultatem sesji Burzy Zdarzeń powinna być lista zdarzeń zgrupowanych w ramach tzw. agregatów dziedzinowych, czyli obiektów realizujących konkretny zbiór logiki biznesowej. W tej postaci łatwo postawić granice między słabo powiązanymi grupami agregatów. Zbiór agregatów po jednej stronie granicy nazywamy Ograniczonym Kontekstem (ang. Bounded Context \cite{boundedcontext})). Są to kandydaci do wydzielenia jako mikroserwisy modelowanej aplikacji.

W ramach pracy inżynierskiej przeprowadzono dla projektowanego systemu sampdzielną sesję Burzy Zdarzeń z wykorzystaniem narzędzia Miro.

Miro \cite{miro} to interaktywna platforma webowa, umożliwiająca tworzenie tablic z notatkami, rysunkami i diagramami. Wybór tego narzędzia był podyktowany jego prostotą użytkowania oraz możliwościami wizualizacji efektów.

W ramach sesji przeprowadzono kolejno następujące kroki:

\begin{itemize}

    \item \textbf{Definicja zdarzeń biznesowych}: Zidentyfikowano kluczowe zdarzenia w procesie biznesowym i zapisuno je na karteczkach (np. "zamówienie złożone", "płatność przyjęta").
    \item \textbf{Układanie zdarzeń}: Zdarzenia zostały ułożone w kolejności chronologicznej, tworząc w ten sposób proces biznesowy.
    \item \textbf{Identyfikacja komend}: Zidentyfikowano komendy, które mogą wywołać zdarzenia (np. "złóż zamówienie").
    \item \textbf{Identyfikacja modeli}: Zidentyfikowano modele danych, które są potrzebne do realizacji procesu biznesowego (np. "zamówienie").
    \item \textbf{Identyfikacja agregatów}: Zidentyfikowano agregaty dziedzinowe, czyli grupy powiązanych ze sobą zdarzeń, komend i modeli danych (np. "zamówienie").
    \item \textbf{Identyfikacja ograniczonych kontekstów}: Zidentyfikowano ograniczone konteksty, czyli grupy powiązanych ze sobą agregatów dziedzinowych (np. "zamówienie").
    \item \textbf{Identyfikacja widoków}: Zidentyfikowano widoki, czyli dane, które są potrzebne do wyświetlenia w interfejsie użytkownika (np. "lista restauracji").

\end{itemize}

\begin{figure}[!h]
    \centering \includegraphics[width=0.8\linewidth]{event_storming2.png}
    \caption{Rezultat sesji Burzy Zdarzeń dla projektowanego systemu}
\end{figure}

\subsection{Identyfikacja Ograniczonych Kontekstów}

W wyniku sesji Burzy Zdarzeń zidentyfikowano pięć kontekstów granicznych, które będą kluczowe dla tworzonej aplikacji. Są to:

\textbf{Restaurant} (\textit{Restauracja}) - obejmuje zarządzanie restauracjami, menu oraz definicją i dostępnością produktów. Potrafi wyliczyć aktualną cenę produktów w zamówieniu oraz zwalidować ich dostępność,

\textbf{Order} (\textit{Zamówienie}) - odpowiada za proces zamówienia, jego składanie, modyfikowanie, anulowanie oraz koordynuje cały proces przygotowania zamówienia do momentu dostawy. Umożliwia również zarządzanie danymi użytkowników i ich adresami dostaw,

\textbf{Delivery} (\textit{Dostawa}) - koncentruje się na logistyce dostaw, monitorowaniu statusu dostawy oraz komunikacji z dostawcą. Zarządza również bazą dostawców i ich lokalizacjami. Odpowiada za proces dopasowania dostawcy do zamówienia,

\textbf{Payment} (\textit{Płatność}) - zarządza procesem płatności, obejmuje różne metody płatności oraz obsługę transakcji. Obsługuje wszelkie rozliczenia w ramach systemu. Odpowiada również za generowanie faktur dla użytkowników, restauracji i dostawców i ich wysyłkę,

\medskip

Powyższe cztery konteksty stały się podstawą do wydzielenia mikroserwisów w ramach projektowanego systemu. Każdy z nich będzie realizowany przez osobny komponent.

\subsection{Wymagania funkcjonalne}

Wymagania funkcjonalne zostały zidentyfikowane na podstawie komend i zdarzeń, które zostały wylistowane w ramach sesji Burzy Zdarzeń. Są to, z podziałem na klasy użytkowników:

\textbf{Jako użytkownik zamawiający}
\begin{itemize}
    \item mogę zarejestrować się w systemie,
    \item mogę dodać detale dostawy,
    \item mogę wylistować restauracje wraz z ich średnią oceną,
    \item mogę rozpocząć proces zamówienia wybierając restaurację,
    \item mogę wybrać dania z menu i dodać je do zamówienia,
    \item mogę złożyć zamówienie,
    \item mogę opłacić zamówienie przy użyciu zewnętrznej bramki płatności,
    \item mogę śledzić status zamówienia,
    \item mogę wylistować historyczne zamówienia,
    \item mogę otrzymać fakturę za zamówienie na podany adres e-mail.
\end{itemize}

\medskip

\textbf{Jako manager restauracji}
\begin{itemize}
    \item mogę zarejestrować restaurację w systemie,
    \item mogę zaktualizować detale i dostępność (otwarta/zamknięta) restauracji,
    \item mogę skonfigurować menu restauracji,
    \item mogę wylistować zamówienia będące w trakcie,
    \item mogę przyjąć albo odrzucić zamówienie,
    \item mogę oznaczyć zamówienie jako gotowe,
    \item mogę wypłacić środki za zamówienia na podany numer konta,
    \item mogę otrzymać fakturę za zamówienie na podany adres e-mail.
\end{itemize}

\medskip

\textbf{Jako dostawca}
\begin{itemize}
    \item mogę zarejestrować się w systemie,
    \item mogę zaktualizować swój status (dostępny/niedostępny),
    \item mogę zobaczyć dostępną ofertę dostawy,
    \item mogę przyjąć albo odrzucić ofertę dostawy,
    \item mogę zaktualizować status dostawy (odebrana z restauracji, dostarczona do użytkownika),
    \item mogę wypłacić środki za dostarczone zamówienia na podany numer konta,
    \item mogę otrzymać fakturę za dostarczone zamówienia na podany adres e-mail.
\end{itemize}

\subsection{Przypadki użycia}

TODO, czy są potrzebne, skoro zostały wylistowane wyżej wymagania?

\subsection{Wymagania niefunkcjonalne}

Wymagania niefunkcjonalne zostały zidentyfikowane m.in. na podstawie ograniczeń biznesowych, wymagań funkcjonalnych oraz doświadczenia Autora w budowaniu skalowalnych aplikacji webowych. Ponadto zostały one zainspirowane listą 12 czynników wpływających na jakość oprogramowania zaproponowaną przez Adama Wigginsa \cite{12factors}.

\textbf{Skalowalność} - każdy z komponentów systemu powinien być skalowalny poziomo w zależności od obciążenia. Każdy mikroserwis powinien móc być replikowalny,

\textbf{Wydajność} - system powinien umożliwiać równoległe przetwarzanie minimum 50 dowolnych żądań HTTP na sekundę,

\textbf{Wysoka dostępność} - możliwość częściowej pracy pomimo utraty niektórych komponentów systemu. System powinien być odporny na awarie,

\textbf{Monitorowanie} - aplikacja powinna umożliwiać monitorowanie i logowanie zdarzeń w celu analizy i debugowania. Każdy mikroserwis powinien logować swoje zdarzenia w centralnym repozytorium, a system powinien umożliwiać monitorowanie stanu mikroserwisów.

\textbf{Testowalność} - system powinien być łatwy do testowania, zarówno jednostkowo jak i integracyjnie oraz wydajnościowo.

\textbf{Bezpieczeństwo} - aplikacja powinna zapewniać odpowiednie zabezpieczenia w celu ochrony poufności, integralności i dostępności danych. Dostęp do mikroserwisów powinien być chroniony przez autoryzację i uwierzytelnianie, a dane powinny być szyfrowane w tranzycie.

\textbf{Synchronizacja i spójność danych} - system powinien zapewniać ostateczną spójność danych pomiędzy mikroserwisami.
\clearpage % Rozdziały zaczynamy od nowej strony.

\section{Architektura systemu}

W tym rozdziale zostanie przedstawiona szczegółowa architektura projektowanego systemu. Celem tego rozdziału jest dostarczenie kompleksowego przeglądu wszystkich kluczowych komponentów, ich wzajemnych zależności, a także sposobu ich integracji.

\subsection{Podział na mikroserwisy}

Część serwerowa systemu została podzielona na mikroserwisy zgodnie z wcześniej wydzielonymi Ograniczonymi Kontekstami. Dodatkowo, poza serwisami związanych z Ograniczonymi Kontekstami, zostały wydzielone serwisy wspólne, które są wykorzystywane przez pozostałe elementy systemu. Są to:

\begin{itemize}

    \item \textbf{Dostawy} (ang. \textit{deliveries}) - implementacja kontekstu "Dostawa",
    \item \textbf{Zamówienia} (ang. \textit{orders}) - implementacja kontekstu "Zamówienie",
    \item \textbf{Płatności} (ang. \textit{payments}) - implementacja kontekstu "Płatność",
    \item \textbf{Restauracje} (ang. \textit{restaurants}) - implementacja kontekstu "Restauracja",
    \item \textbf{Zapytania} (ang. \textit{queries}) - serwis odpowiedzialny za budowanie bazodanowych projekcji danych oraz obsługę zapytań. Jest wykorzystywany przez wszystkie pozostałe serwisy oraz część kliencką systemu,
    \item \textbf{Sagi} (ang. \textit{sagas}) - serwis odpowiedzialny za obsługę/orkiestrację długo trwających procesór biznesowych. Implementuje wzorzec Saga. Jest wykorzystywany przez wszystkie pozostałe serwisy.

\end{itemize}

Wydzielono również moduł niebędący mikroserwisem, a jedynie biblioteką, która jest wykorzystywana przez wszystkie pozostałe serwisy. Jest to \textbf{Wspólny kod} (ang. \textit{shared}), który zawiera wspólne dla wszystkich serwisów klasy, interfejsy, konfiguracje, itp.

\subsection{Model C4 systemu}

W celu przedstawienia architektury systemu w sposób zrozumiały i przejrzysty, został wykorzystany model C4. Jest to zbiór abstrakcyjnych diagramów, które pozwalają na przedstawienie architektury systemu na różnych poziomach szczegółowości. W tym rozdziale zostaną przedstawione diagramy dla poziomów kontekstu, kontenerów oraz komponentów.

TODO

\subsection{Protokoły i formaty komunikacji}

W celu zapewnienia komunikacji pomiędzy poszczególnymi komponentami systemu, zostały wykorzystane następujące protokoły i formaty komunikacji:

TODO przeczytac bo glupoty pisze

\begin{itemize}

    \item \textbf{Protokół HTTP} (ang. \textit{Hypertext Transfer Protocol}) - protokół warstwy aplikacji, który jest wykorzystywany do komunikacji pomiędzy częścią kliencką systemu a serwisami. Jest to protokół bezstanowy, który wykorzystuje metody takie jak GET, POST, PUT, DELETE, itp. do przesyłania danych. W projekcie został wykorzystany do komunikacji pomiędzy częścią kliencką systemu a serwisami, a także pomiędzy serwisami. W celu zapewnienia bezpieczeństwa komunikacji, został wykorzystany protokół HTTPS, który jest szyfrowaną wersją protokołu HTTP,

    \item \textbf{REST} (ang. \textit{Representational State Transfer}) - styl architektoniczny, który jest wykorzystywany do projektowania rozproszonych systemów. W projekcie został wykorzystany do komunikacji pomiędzy częścią kliencką systemu a serwisami, a także pomiędzy serwisami. REST definiuje zestaw zasad, które określają jak powinny wyglądać interfejsy komunikacyjne pomiędzy komponentami systemu.
    
    \item \textbf{Protokół gRPC} (ang. \textit{Remote Procedure Call}) - otwarty protokół zdalnego wywoływania procedur (RPC) opracowany przez Google. Używa on formatu przesyłania danych Protocol Buffers do efektywnego serializowania struktur danych, oferując wydajne, skalowalne i językowo niezależne mechanizmy komunikacji między usługami. W projekcie został wykorzystany do komunikacji pomiędzy serwisami a kolejką komunikatów oraz magazynem zdarzeń.

    \item \textbf{Format JSON} (ang. \textit{JavaScript Object Notation}) - format danych, który jest wykorzystywany do przesyłania danych pomiędzy komponentami systemu. Jest to format tekstowy, który jest niezależny od języka programowania. W projekcie został wykorzystany do przesyłania danych pomiędzy częścią kliencką systemu a serwisami.
    
    \item \textbf{Format XML} (ang. \textit{Extensible Markup Language}) - format danych, który jest wykorzystywany do przesyłania danych pomiędzy komponentami systemu. Jest to format tekstowy, który jest niezależny od języka programowania. W projekcie został wykorzystany do przesyłania danych pomiędzy serwisami, w postaci zdarzeń.

\end{itemize}

\subsection{Persystencja danych}

TODO magazyn zdarzeń

TODO projekcyjna baza danych

\subsection{Kolejka komunikatów i magazyn zdarzeń}

TODO axon server

\subsection{Uwierzytelnianie i autoryzacja}

TODO Auth0

TODO OAuth2 (Authorization code grant)

TODO JWT

\subsection{Integracje zewnętrzne}

TODO Google Maps Platform

TODO Stripe

TODO Mailjet

TODO invoice generator
\clearpage % Rozdziały zaczynamy od nowej strony.

\section{Implementacja}

Rozdział ten ma na celu szczegółowy opis implementacji systemu. Zostaną w nim przedstawione wykorzystane narzędzia, podział na moduły oraz szczegóły implementacyjne. Swoje miejsce znajdą tutaj również fragmenty kodu źródłowego, które mają na celu ułatwić zrozumienie sposobu działania systemu.

\subsection{Część serwerowa}

Część serwerową systemu stanowi aplikacja o architekturze mikroserwisowej, napisana w języku Kotlin z wykorzystaniem frameworków Spring Boot oraz Axon Framework.

\subsubsection{Użyte narzędzia}

W części serwerowej aplikacji wykorzystano następujące najważniejsze narzędzia:

\textbf{Kotlin} \cite{kotlin} to język programowania stworzony przez firmę JetBrains, działający na maszynie wirtualnej Javy (JVM) \cite{jvm}. Kotlin jest językiem statycznie typowanym, który łączy w sobie cechy zarówno języków obiektowych, jak i funkcyjnych. Jest on kompilowany do kodu bajtowego Javy, a jego składnia jest w dużej mierze zgodna z Javą, co czyni go łatwym do nauki i zrozumienia dla programistów Javy. Kotlin jest językiem wieloplatformowym, co oznacza, że może być kompilowany do kodu bajtowego Javy, kodu bajtowego Javy na Androida, kodu JavaScript oraz kodu natywnego. Kotlin jest językiem ogólnego przeznaczenia, który może być wykorzystywany do tworzenia aplikacji webowych, mobilnych, desktopowych, a nawet do tworzenia skryptów. Jego jedną z ważniejszych zalet jest wprowadzenie nullowalności na poziomie systemu typów, co pozwala na wykrywanie błędów związanych z niepoprawnym użyciem wartości null w czasie kompilacji, a nie w czasie działania programu.

\textbf{Spring Boot} \cite{springboot} to framework w ekosystemie Spring, który ułatwia tworzenie i rozwijanie aplikacji webowych oraz mikroserwisów w językach Java i Kotlin. Spring Boot oferuje wsparcie dla mikroserwisów, ułatwiając ich tworzenie, testowanie i wdrażanie, co czyni go popularnym wyborem wśród programistów pracujących nad nowoczesnymi, skalowalnymi aplikacjami.

Spring Boot ma silne powiązanie z koncepcją "cloud native", która odnosi się do projektowania aplikacji specjalnie na potrzeby chmury. Spring Boot ułatwia tworzenie aplikacji mikroserwisowych, które są nieodzownym elementem architektury cloud native, dostarczając funkcjonalności takie jak łatwa integracja z kontenerami (np. Docker), obsługa konfiguracji zewnętrznej, zarządzanie usługami przez service discovery oraz wspieranie wzorców takich jak circuit breaker. Te cechy sprawiają, że Spring Boot jest idealnym wyborem dla tworzenia aplikacji przygotowanych do działania w środowiskach chmurowych, oferujących skalowalność, elastyczność i odporność.

\textbf{Axon Framework} to framework do tworzenia aplikacji w architekturze opartej na zdarzeniach (event-driven) i wzorcu CQRS. Wspiera również DDD, Event Sourcing oraz wzorzec Saga. Axon Framework jest napisany w języku Java, ale może być wykorzystywany również w języku Kotlin. Axon Framework dostarcza abstrakcje do tworzenia aplikacji opartych na zdarzeniach, takich jak agregaty, komendy, zdarzenia, szyna wiadomości, sagi, itp. Axon Framework jest narzędziem open source, rozwijanym przez firmę AxonIQ.

\textbf{Axon Server} jest infrastrukturalnym elementem ekosystemu Axon, pełniącym rolę kolejki komunikatów oraz magazynu zdarzeń. Oferuje on również narzędzia do monitoringu i zarządzania aplikacjami opartymi na Axon Framework.

\subsubsection{Architektura heksagonalna}

Część serwerowa aplikacji została zaimplementowana zgodnie z architekturą heksagonalną (ang. hexagonal architecture), która jest jedną z popularnych architektur aplikacji serwerowych, wykorzystywanych przy złożonych projektach informatycznych. Architektura ta jest również znana pod nazwą architektury czystej (ang. clean architecture) lub architektury portów i adapterów (ang. ports and adapters architecture). Została ona zaproponowana przez Alistaira Cockburna w 2005 roku \cite{cockburn2005hexagonal}.

Architektura heksagonalna jest architekturą warstwową, która składa się z trzech warstw: warstwy adapterów, warstwy dziedziny oraz warstwy infrastruktury. Warstwa dziedziny jest główną warstwą aplikacji, która zawiera logikę biznesową. Warstwa adapterów jest warstwą zewnętrzną, która zawiera adaptery wejściowe i wyjściowe, które są odpowiedzialne za komunikację z zewnętrznymi systemami. Warstwa infrastruktury jest warstwą wewnętrzną, która zawiera implementację adapterów wejściowych i wyjściowych. Warstwa infrastruktury jest odpowiedzialna za konfigurację aplikacji oraz za integrację z zewnętrznymi systemami (np. bazą danych, systemem plików, itp.).

\subsubsection{Podział na pakiety}

Z powodu zastosowania architektury heksagonalnej, część serwerowa aplikacji została podzielona na pakiety zgodnie z warstwami tejże architektury. Podział na pakiety przykładowego serwisu został przedstawiony na wycinku \ref{lst:server-packages}.

\begin{lstlisting}[caption={Podział na pakiety części serwerowej projektu},label={lst:server-packages},captionpos=b]
- adapter/
    - in/
    - out/
- domain/
    - query/
    - command/
- config/
- infrastructure/
- Application.kt
\end{lstlisting}

W kolejnych podrozdziałach zostaną przedstawione szczegóły implementacyjne poszczególnych warstw wraz z ich opisem.

\subsubsection{Warstwa adapterów wejściowych} 

Warstwa ta obejmuje pakiet \textit{adapter.in}. Zawiera ona tzw. adaptery wejściowe, czyli komponenty odpowiedzialne za rozpoczynanie przepływu sterowania w aplikacji. Adaptery wejściowe są odpowiedzialne za obsługę zapytań i komend, które są wysyłane do aplikacji przez użytkowników lub inne systemy. W implementowanym systemie adapterami wejściowymi są kontrolery REST API oraz serwisy nasłuchujące na zdarzenia z kolejki komunikatów.

Na wycinku \ref{lst:server-in-adapter} przedstawiono przykładowy kod kontrolera REST API, który obsługuje zapytanie HTTP POST na ścieżce \textit{/api/v1/orders}. Zapytanie to rozpoczyna zamówienie ze strony użytkownika zamawiającego. W ciele zapytania znajduje się obiekt JSON, który jest mapowany na obiekt klasy \textit{StartOrderRequest}. Obiekt ten jest następnie przekazywany do komponentu \textit{ReactorCommandGateway}, który jest odpowiedzialny za wysłanie komendy \textit{StartOrderCommand} do szyny komend. Komenda ta jest następnie przekazywana do odpowiedniego agregatu dziedzinowego, który jest odpowiedzialny za jej obsługę. Rezultatem synchronicznym obsługi komendy jest nadany identyfikator zamówienia, w postacu UUID, który jest zwracany w odpowiedzi HTTP. 

\begin{lstlisting}[caption={Kod kontrolera REST API obsługującego zapytania dotyczące zamówień},label={lst:server-in-adapter},captionpos=b,language=Kotlin,numbers=left]
    @RestController
    @RequestMapping("/api/v1/orders")
    class OrderController(
        private val reactorCommandGateway: ReactorCommandGateway
    ) {
        @PostMapping
        @ResponseStatus(HttpStatus.CREATED)
        @PreAuthorize("hasAnyAuthority('${Scopes.ORDER.WRITE}')")
        fun startOrder(
            @RequestBody request: StartOrderRequest,
            exchange: ServerWebExchange
        ): Mono<UuidWrapper> {
            val command = mapToStartOrderCommand(request, exchange)
            val id = reactorCommandGateway.send<UUID>(command)
            return id.map { UuidWrapper(it) }
        }
    }
\end{lstlisting}

Kod kontrolera został napisany z wykorzystaniem biblioteki programowania reaktywnego Spring WebFlux, która pozwala na obsługę żądań HTTP w sposób asynchroniczny. Pozwala to na obsługę większej liczby żądań przy użyciu mniejszej liczby wątków, co przekłada się na większą wydajność aplikacji.

Poza kodem obsługującym żądanie HTTP, kontroler zawiera również adnotacje, które definiują uprawnienia wymagane do wykonania żądania. W tym przypadku żądanie wymaga posiadania uprawnień \textit{ORDER.WRITE}, które są zdefiniowane w klasie \textit{Scopes}. Jest to część biblioteki Spring Security, implementującej standard OAuth 2.0.

\subsubsection{Warstwa adapterów wyjściowych} 

Warstwa ta obejmuje pakiet \textit{adapter.out}. Zawiera ona tzw. adaptery wyjściowe, czyli komponenty odpowiedzialne za komunikację z zewnętrznymi systemami. W implementowanym systemie adapterami wyjściowymi są komponenty oferujące dostęp do bazy danych oraz komunikację z zewnętrznymi systemami np. Google Maps Platform.

Na wycinku \ref{lst:server-out-adapter} przedstawiono przykładowy kod komponentu oferującego dostęp do bazy danych. Jest to implementacja (adapter) interfejsu (portu) \textit{OrderDeliveryRepository}, który jest odpowiedzialny za zapisywanie i odczytywanie encji \textit{OrderDeliveryEntity} z bazy danych. Implementacja ta wykorzystuje repozytorium Spring Data, które jest komponentem oferującym dostęp do bazy MongoDB. Repozytorium to jest wstrzykiwane do klasy \textit{DbOrderDeliveryRepository} przy pomocy mechanizmu wstrzykiwania zależności oferowanego przez framework Spring. 

\begin{lstlisting}[caption={Kod implementacji repozytorium dziedzinowego projekcji Zamówienia},label={lst:server-out-adapter},captionpos=b,language=Kotlin,numbers=left]
    @Service
    class DbOrderDeliveryRepository(
        private val repository: SpringDataOrderDeliveryRepository
    ) : OrderDeliveryRepository {
        override fun save(delivery: OrderDeliveryEntity) {
            repository.save(delivery).block()
        }
    
        override fun load(id: UUID): OrderDeliveryEntity? {
            return repository.findById(id).block()
        }
    
        override fun loadOffers(): List<OrderDeliveryEntity> {
            return repository
                .findAllByStatus(DeliveryStatus.OFFER)
                .collectList().block() ?: listOf()
        }
    }
\end{lstlisting}

\subsubsection{Warstwa dziedziny} 

Warstwa ta obejmuje pakiet \textit{domain}. Zawiera ona komponenty, które są odpowiedzialne za logikę biznesową aplikacji. W implementowanym systemie komponentami tymi są agregaty dziedzinowe, komendy, zdarzenia, porty, sagi, encje bazodanowe, klasy wyjątków, zapytania, komponenty budujące projekcje oraz obiekty transferu danych.

Warstwa dziedziny jest podzielona zgodnie ze wzorcem CQRS na część komend oraz część zapytań. Część komend jest zawarta w podpakiecie \textit{command}, a część zapytań w podpakiecie \textit{query}.

Na wycinku \ref{lst:server-domain} przedstawiono częściowy kod komponentów warstwy dziedziny odpowiadającej za rozpoczynanie zamówienia z perspektywy użytkownika systemu. Są to: agregat dziedzinowy \textit{Order}, komenda \textit{StartOrderCommand} oraz zdarzenie \textit{OrderStartedEvent}. Przedstawiono również zawartość portu \textit{OrderVerificationPort}, który jest odpowiedzialny za wstępną walidację zamówienia.

\begin{lstlisting}[caption={Kod komponentów odpowiedzialnych za rozpoczynanie zamówień},label={lst:server-domain},captionpos=b,language=Kotlin,numbers=left]
    @Aggregate
    internal class Order {
    
        @AggregateIdentifier
        private lateinit var id: UUID
        private lateinit var status: OrderStatus
    
        @AggregateMember
        private val items: MutableMap<UUID, Int> = mutableMapOf()
    
        @CommandHandler
        constructor(
            command: StartOrderCommand,
            orderVerificationPort: OrderVerificationPort
        ) {
            require(
                orderVerificationPort.restaurantExists(command.restaurantId)
            )
            require(
                orderVerificationPort.isRestaurantOpen(command.restaurantId)
            )
    
            apply(
                OrderStartedEvent(
                    orderId = command.orderId,
                    restaurantId = command.restaurantId,
                    userId = command.userId,
                    status = OrderStatus.CREATED
                )
            )
        }
    }

    data class StartOrderCommand(
        @TargetAggregateIdentifier val orderId: UUID,
        val userId: String,
        val restaurantId: UUID
    )

    data class OrderStartedEvent(
        val orderId: UUID,
        val restaurantId: UUID,
        val userId: String,
        val status: OrderStatus
    )

    interface OrderVerificationPort {
        fun restaurantExists(restaurantId: UUID): Boolean

        fun isRestaurantOpen(restaurantId: UUID): Boolean
    }
\end{lstlisting}

Kod ten jest wywoływany w momencie otrzymania komendy \textit{StartOrderCommand} przez agregat \textit{Order}. Komenda ta jest wysyłana do agregatu przez adapter wejściowy, zaprezentowany w rozdziale 5.1.4. Agregat \textit{Order} jest odpowiedzialny za wstępną walidację zamówienia, która jest realizowana przy pomocy portu \textit{OrderVerificationPort}. W przypadku tej komendy, system weryfikuje czy podana resutacja istnieje i czy jest otwarta. Port walidacyjny jest interfejsem, implementowanym przez odpowiedni adapter wyjściowy.

W przypadku, gdy wstępna walidacja zamówienia zakończy się sukcesem, agregat \textit{Order} emituje zdarzenie \textit{OrderStartedEvent}, które jest zapisywane w magazynie zdarzeń, kontrolowanym przez Axon Server. Na zdarzenie nasłuchują inne komponenty systemu, które mogą na jego podstawie budować projekcje, wysyłać powiadomienia do użytkowników, lub podejmować inne akcje.

Dzięki zastosowaniu Axon Framework nie ma konieczności pisania kodu odpowiedzialnego za ładowanie agregatów dziedzinowych, wywoływanie odpowiednich metod obsługujących komendy czy publikowanie zdarzeń. Większość interakcji odbywa się bez udziału programisty, przy pomocy odpowiednich adnotacji, takich jak np. \textit{@Aggregate}, \textit{@AggregateIdentifier}, \textit{@CommandHandler} itp.

\subsubsection{Warstwa prezentacji/zapytań}

Warstwa ta obejmuje pakiet \textit{query}. Zawiera ona komponenty, które są odpowiedzialne za obsługę zapytań użytkowników systemu. W implementowanym systemie komponentami tymi są procesory zdarzeń budujące projekcje, zapytania oraz klasy pomocnicze służące do persystencji danych w bazie danych.

Na wycinku \ref{lst:server-domain-query} przedstawiono kod procesora zdarzeń budującego projekcję zamówienia. Procesor ten nasłuchuje na zdarzenie \textit{OrderStartedEvent} i na jego podstawie buduje projekcję zamówienia, która jest zapisywana w bazie danych. Procesor ten jest wywoływany automatycznie przez Axon Framework, w momencie otrzymania odpowiednigo zdarzenia.

\begin{lstlisting}[caption={Kod procesora zdarzeń budującego projekcję zamówienia},label={lst:server-domain-query},captionpos=b,language=Kotlin,numbers=left,showstringspaces=false]
    @ProcessingGroup("projection")
    class OrderHandler(
        private val orderRepository: OrderRepository,
        private val restaurantRepository: RestaurantRepository
    ) {
    
        @EventHandler
        fun on(event: OrderStartedEvent) {
            val restaurant = restaurantRepository.load(
                event.restaurantId
            )
    
            val restaurantLocation = restaurant?.location 
                ?: error("Restaurant not found")
    
            val entity = OrderEntity(
                id = event.orderId,
                restaurantId = event.restaurantId,
                restaurantLocation = restaurantLocation,
                userId = event.userId,
                status = OrderStatus.CREATED,
                items = mutableMapOf(),
                createdAt = Instant.now()
            )
            orderRepository.save(entity)
        }
    }
\end{lstlisting}

\subsubsection{Warstwa konfiguracji}

Warstwa ta obejmuje pakiet \textit{config}. Zawiera ona komponenty, które są odpowiedzialne za konfigurację aplikacji.

Przykładowym komponentem w implementowanym systemie jest filtr zapytań, odpowiedzialny za odczytywanie atrybutu \textit{subject} z tokenu JWT oraz zapisywanie go w kontekście żądania. Jest to używane m.in. do poprawnej autoryzacji użytkownika w momencie wywoływania komendy (np. dostęp do zasobu ma tylko jego autor). Kod tego komponentu został przedstawiony na wycinku \ref{lst:server-config}.

\begin{lstlisting}[caption={Kod filtra zapytań},label={lst:server-config},captionpos=b,language=Kotlin,numbers=left,showstringspaces=false]
    @Component
    @Profile("!nosecurity")
    class UserContextWebFilter : WebFilter {
        override fun filter(
            exchange: ServerWebExchange,
            chain: WebFilterChain
        ): Mono<Void> {
            return ReactiveSecurityContextHolder.getContext()
                .mapNotNull { it.authentication.principal }
                .cast(Jwt::class.java)
                .doOnNext {
                    val userId = it.subject.removePrefix("auth0|")
                    exchange.setUserId(userId)
                }
                .then(chain.filter(exchange))
        }
    }
\end{lstlisting}

Dzięki zastosowaniu adnotacji \textit{@Profile("!nosecurity")} możliwe jest wyłączenie tego filtra w trybie \textit{nosecurity}, który jest używany w testach wydajnościowych, w których wyłączone są mechanizmy uwierzytelniania i autoryzacji.

\subsubsection{Warstwa infrastruktury}

Warstwa ta obejmuje pakiet \textit{infrastructure}. Zawiera ona kod, niezwiązany z domeną aplikacji, który jest odpowiedzialny za integrację z konkretnymi technologiami, np. bazą danych, systemem plików, itp. W implementowanym systemie komponentami tej warstwy są np. implementacje repozytoriów bazodanowych używające konkretnej technologii - Spring Data MongoDB. Są to również komponenty służące do komunikacji z zewnętrznymi usługami, np. klient Google Maps Platform.

Na wycinku \ref{lst:server-infrastructure} przedstawiono przykładowy kod implementacji repozytorium bazodanowego. Jest to implementacja interfejsu \textit{ReactiveMongoRepository}, który jest komponentem oferującym dostęp do bazy MongoDB. Mimo faktu, że repozytorium to jest interfejsem, a nie klasą, to dzięki mechanizmowi refleksji oferowanemu przez framework Spring, implementacja ta jest automatycznie generowana w czasie uruchamiania aplikacji.

\begin{lstlisting}[caption={Kod implementacji repozytorium bazodanowego},label={lst:server-infrastructure},captionpos=b,language=Kotlin,numbers=left,showstringspaces=false]
    @Repository
    interface SpringDataOrderRepository 
        : ReactiveMongoRepository<OrderEntity, UUID> {
    
        fun findAllByUserId(userId: String): Flux<OrderEntity>
    }
\end{lstlisting}

Repozytorium bazowe, \textit{ReactiveMongoRepository} zawiera podstawowe metody do operacji na bazie danych, takie jak np. \textit{save}, \textit{findById}, \textit{findAll}, itp. Repozytorium \textit{SpringDataOrderRepository} rozszerza repozytorium bazowe i dodaje do niego dodatkową metodę \textit{findAllByUserId}, która zwraca wszystkie zamówienia danego użytkownika.

\subsubsection{Współdzielony kod}

Kod współdzielony przez różne serwisy w ramach systemu został umieszczony w osobnym module \textit{shared}. Zawiera on m.in. klasy reprezentujące obiekty transferu danych, wyjątki, stałe, wspólną konfigurację itp.

Moduł ten jest wersjonowany i publikowany jako artefakt systemu budowania Maven, dzięki czemu może być umieszczony w centralnym repozytorium i importowany przez inne projekty.

Przykład współdzielonej klasy zaprezentowany jest na listingu \ref{lst:server-shared}

\begin{lstlisting}[caption={Kod klasy konfigurującej mechanizm uwierzytelniania i autoryzacji},label={lst:server-shared},captionpos=b,language=Kotlin,numbers=left,showstringspaces=false]
    @Configuration
    @EnableReactiveMethodSecurity
    @EnableWebFluxSecurity
    @Profile("!nosecurity")
    class SecurityConfig {
    
        @Bean
        fun springSecurityFilterChain(
            http: ServerHttpSecurity
        ): SecurityWebFilterChain {
            return http.authorizeExchange {
                it.pathMatchers("/actuator/**").permitAll()
                it.pathMatchers("/api/v1/payments/stripe-webhook").permitAll()
                it.anyExchange().authenticated()
            }.csrf {
                it.disable()
            }.cors {
                it.disable()
            }.oauth2ResourceServer {
                it.jwt {}
            }.build()
        }
    }
\end{lstlisting}

Klasa ta jest odpowiedzialna za konfigurację mechanizmu uwierzytelniania i autoryzacji. Została ona umieszczona w module \textit{shared}, ponieważ jest ona wspólna dla wszystkich serwisów w ramach systemu. Uruchamia ona mechanizm uwierzytelniania i autoryzacji oferowany przez framework Spring Security. Uwierzytelnianie odbywa się przy pomocy tokenów JWT. Dodatkowo, wyłącza ona mechanizmy obrony przed atakiem CSRF oraz CORS, które są niepotrzebne w przypadku zastosowanej infrastruktury sieciowej systemu.

Domyślnie, żądania na wszystkie ścieżki aplikacji wymagają uwierzytelniania. Są jednak dwie ścieżki, gdzie takie mechanizmy są wyłączone. Pierwsza z nich to ścieżka \textit{/actuator/**}, która jest wykorzystywana przez framework Spring Boot do oferowania statusu aplikacji, wykorzystywanego m.in. przez orkiestratora kontenerów Kubernetes. Druga ścieżka to \textit{/api/v1/payments/stripe-webhook}, która jest wykorzystywana przez serwis odpowiedzialny za płatności, do obsługi webhooków z serwisu Stripe.

\subsection{Część kliencka}

Część kliencka aplikacji została zaimplementowana jako aplikacja przeglądarkowa typu Multi Page Application (MPA) w języku TypeScript z wykorzystaniem frameworków React oraz Remix.

TODO czemu nie SPA?

\subsubsection{Użyte narzędzia}

\subsubsection{Podział na katalogi}

\subsubsection{Komponenty interfejsu użytkownika}

\subsubsection{Komunikacja z częścią serwerową}

\subsubsection{Obsługa ścieżek}

\subsubsection{Współdzielony kod}

\subsubsection{Przechowywanie stanu aplikacji}
\clearpage % Rozdziały zaczynamy od nowej strony.

\section{Wdrożenie}

XXX

\subsection{Klaster Kubernetes}

\subsection{Axon Server}

\subsection{Google Cloud Platform}
\clearpage % Rozdziały zaczynamy od nowej strony.

\section{Testowanie systemu}

Aby zapewnić najwyższą jakość tworzonego oprogramowania, kluczowe jest włączenie etapu testowania do procesu jego rozwoju. Celem testowania jest identyfikacja i eliminacja błędów, zazwyczaj przed oficjalnym wdrożeniem produktu. W tym procesie stosuje się różnorodne metody i narzędzia. We wdrożonym systemie, zostały wykorzystane cztery główne rodzaje testów: testy jednostkowe, które koncentrują się na najmniejszych częściach kodu; testy integracyjne, sprawdzające współdziałanie poszczególnych modułów; testy akceptacyjne, które oceniają oprogramowanie z perspektywy użytkownika końcowego; oraz testy wydajnościowe, mające na celu ocenę szybkości i stabilności działania systemu.

\subsection{Testy jednostkowe}

Testy jednostkowe są najmniejszymi testami, które można przeprowadzić na oprogramowaniu. Ich celem jest sprawdzenie, czy najmniejsze części kodu, np. pojedyncze funkcje, klasy, działają poprawnie. W ramach pracy inżynierskiej testy jednostkowe zostały przeprowadzone przy pomocy narzędzia Kotest \cite{kotest} oraz pakietu \textit{org.axonframework:axon-test} pomocniczego w testowaniu aplikacji opartych na frameworku Axon.

Pokrycie kodu reprezentowane jest przez procentowy udział linii kodu, które zostały przetestowane. W ramach pracy inżynierskiej, pokrycie kodu zostało zmierzone przy pomocy narzędzia JaCoCo \cite{jacoco}. Cel pokrycia kodu został ustalony na poziomie 80\%. W implementowanym systemie, pokrycie całego kodu wyniosło 83\%, a pokrycie kodu warstwy dziedziny wyniosło 100\%.

\begin{lstlisting}[caption={Przykładowy test jednostkowy serwisu zamówień},label={lst:testing-unit},captionpos=b,language=Kotlin,numbers=left]
class OrderTest {

    private lateinit var testFixture: AggregateTestFixture<Order>
    private lateinit var orderVerificationPort: OrderVerificationPort

    @BeforeEach
    fun setUp() {
        orderVerificationPort = mockk<OrderVerificationPort>()
        testFixture = AggregateTestFixture(Order::class.java)
        testFixture.registerInjectableResource(orderVerificationPort)
    }

    @Test
    fun `given created order, should finalize order`() {
        val orderStartedEvent = OrderStartedEvent(
            orderId = UUID.randomUUID(),
            userId = "user",
            restaurantId = UUID.randomUUID(),
            status = OrderStartedEvent.OrderStatus.CREATED
        )

        val orderItemAddedEvent = OrderItemAddedEvent(
            orderId = orderStartedEvent.orderId,
            productId = UUID.randomUUID()
        )

        val finalizeOrderCommand = FinalizeOrderCommand(
            orderId = orderStartedEvent.orderId,
            userId = orderStartedEvent.userId
        )

        every {
            orderVerificationPort.productExists(
                orderItemAddedEvent.productId,
                orderStartedEvent.restaurantId
            )
        } returns true

        val orderFinalizedEvent = OrderFinalizedEvent(
            orderId = orderStartedEvent.orderId,
            userId = orderStartedEvent.userId,
            restaurantId = orderStartedEvent.restaurantId,
            items = mapOf(orderItemAddedEvent.productId to 1)
        )

        testFixture.given(orderStartedEvent, orderItemAddedEvent)
            .`when`(finalizeOrderCommand)
            .expectSuccessfulHandlerExecution()
            .expectEvents(orderFinalizedEvent)
    }
}
\end{lstlisting}

Na listingu \ref{lst:testing-unit} przedstawiono przykładowy test jednostkowy serwisu zamówień, weryfikujący zachowanie agregatu dla komendy finalizującej zamówienie. W liniach 6-11 przedstawiono przygotowanie testu, polegające na utworzenie obiektu testowego oraz zarejestrowanie w nim testowej implementacji portu weryfikacji zamówień. W liniach 13-51 przedstawiono scenariusz testowy, polegający na wywołaniu komendy finalizującej zamówienie, a następnie sprawdzeniu, czy agregat wyemitował oczekiwane zdarzenie.

Dzięki pakietowi pomocniczemu frameworka Axon takie testy są łatwe w implementacji i dalszym utrzymaniu. Dzięki zastosowaniu technik DDD oraz Event Sourcing możliwe jest przetestowanie zachowania logiki biznesowej bez konieczności uruchamiania całej aplikacji, przez co wykonywanie takich testów jest szybkie i wydajne.

\subsection{Testy integracyjne}

Testy integracyjne sprawdzają współdziałanie poszczególnych modułów systemu. W ramach pracy inżynierskiej testy jednostkowe zostały przeprowadzone przy pomocy tych samych narzędzi co jednostkowe, z dodatkową pomocą biblioteki \linebreak \textit{org.springframework.boot:spring-boot-starter-test}, pozwalającej na uruchomienie testów w kontekście aplikacji Spring Boot.

W celu przygotowania środowiska testowego należy uruchomić lokalnie instancję Axon Server np. przy pomocy Dockera \cite{docker}.

\begin{lstlisting}[caption={Przykładowy test integracyjny serwisu zamówień},label={lst:testing-integration},captionpos=b,language=Kotlin,numbers=left]
@SpringBootTest
@ActiveProfiles("test")
class OrderTestIT {

    @Autowired
    private lateinit var commandGateway: CommandGateway

    @Autowired
    private lateinit var aggregateRepository: Repository<Order>

    @Test
    fun `given open restaurant and user, should start order`() {
        val orderId = commandGateway.sendAndWait<UUID>(
            StartOrderCommand(
                orderId = UUID.randomUUID(),
                userId = "userId",
                restaurantId = UUID.randomUUID()
            )
        )

        val order = aggregateRepository.load(orderId.toString())
        order.userId shouldBe "userId"
    }
}
\end{lstlisting}

Na listingu \ref{lst:testing-integration} przedstawiono przykładowy test integracyjny serwisu zamówień, weryfikujący, że po wysłaniu komendy rozpoczęcia zamówienia, agregat został poprawnie utworzony. W przeciwieństwie do testów jednostkowych test ten weryfikuje również działanie komponentów odpowiedzialnych za przesyłanie komend do agregatów oraz zapisywanie zdarzeń do magazynu zdarzeń. Testy te są bardziej złożone w implementacji, ale pozwalają na weryfikację działania całego systemu.

\subsection{Testy akceptacyjne}

Testy akceptacyjne sprawdzają, czy system spełnia wymagania użytkownika końcowego. W ramach pracy inżynierskiej testy akceptacyjne zostały przeprowadzone przy pomocy narzędzia RestAssured \cite{restassured}, które pozwala na testowanie interfejsu REST API. Zaimplementowano jeden scenariusz testowy, który przechodzi przez wszystkie funkcjonalności systemu, od rejestracji użytkownika, przez dodanie restauracji, złożenie zamówienia, aż do jego dostarczenia.

Służy on również jako weryfikacja poprawnego działania systemu po wdrożeniu nowej wersji w ramach potoku CI/CD.

\begin{lstlisting}[caption={Scenariusz testowy w ramach testów akceptacyjnych},label={lst:testing-acceptance},captionpos=b,language=Kotlin,numbers=left]
@Test
fun e2e() {
    // setup
    setupRestaurant()
    setupOrderingUser()
    setupCourier()

    // user
    createOrder()
    payOrder()

    // restaurant
    acceptRestaurantOrder()
    prepareRestaurantOrder()

    // courier
    acceptDeliveryOffer()
    pickupDelivery()
    deliverDelivery()

    // payments
    withdrawRestaurantBalance()
    withdrawCourierBalance()
}
\end{lstlisting}

Na listingu \ref{lst:testing-acceptance} przedstawiono jedynie fragment wspomnianego testu, ze względu na jego długość. Test ten weryfikuje, czy system spełnia wymagania użytkownika końcowego, a także czy wszystkie komponenty współpracują ze sobą poprawnie. Test ten jest bardzo złożony w implementacji, ale pozwala na weryfikację działania całego systemu.

Weryfikuje on również integrację z zewnętrznymi systemami, takimi jak system płatności Stripe, poprzez wykorzystanie biblioteki Selenium \cite{selenium}, umożliwiającej automatyzację przeglądarek internetowych, co pozwala na manualne przejście procesu płatności.

\subsection{Testy wydajnościowe}

Testy wydajnościowe mają na celu ocenę szybkości i stabilności działania systemu. Pozwalają one przewidywać zachowanie systemu pod wzrastającym obciążeniem, umożliwiając zaplanowanie odpowiednich ulepszeń infrastruktury i kodu.

W ramach pracy inżynierskiej przeprowadzono testy wydajnościowe przy pomocy narzędzia Locust \cite{locust} w języku Python. Narzędzie to pozwala na symulację wielu użytkowników, którzy wykonują określone akcje w systemie. Przeprowadzono testy z perspektywy trzech rodzajów użytkowników: użytkownika zamawiającego, managera restauracji oraz kuriera. Każdy rodzaj użytkownika został zdefiniowany przez klasę w języku Python, która implementuje logikę wykonywania akcji w systemie. 

Przykład scenariusza testowego został przedstawiony na listingu \ref{lst:testing-performance}.

\begin{lstlisting}[caption={Scenariusz testowy w ramach testów wydajnościowych obsługujących managera restauracji},label={lst:testing-performance},captionpos=b,language=Python,numbers=left]
class RestaurantManager(HttpUser):
    def on_start(self):
        self.client.headers["X-User-Id"] = fake.uuid4()
        self._create_restaurant()

    @task
    def e2e(self):
        restaurant_orders = self.client.get(
            f"/v2/restaurants/{self.restaurant_id}/orders"
        ).json()

        selected_order = random.choice(restaurant_orders)
        restaurant_order_id = selected_order.get("restaurantOrderId")

        self.client.put(f"/v1/orders/{restaurant_order_id}/accept")

        self.client.put(f"/v1/orders/{restaurant_order_id}/prepare" )
\end{lstlisting}

Klasa ta imituje zachowanie managera restauracji, który po zalogowaniu się do systemu, pobiera listę zamówień, wybiera jedno z nich, a następnie akceptuje i przygotowuje je do dostarczenia.

W ramach testów wydajnościowych przeprowadzono testy z perspektywy 100, 200, 300, 500 i 1000 użytkowników. Wszystkie testy zostały wykonane na lokalnym środowisku, przy pomocy trzech instancji narzędzia Locust. Wszystkie testy zostały przeprowadzone na wdrożonej w chmurowym klastrze Kubernetes produkcyjnej wersji systemu, bez żadnych dodatkowych optymalizacji. Klaster był skonfigurowany z trzema węzłami typu \textit{e2-standard-4}, jak opisano w rozdziale dotyczącym wdrożenia. Każdy serwis systemu został uruchomiony w czterech instancjach. Liczba użytkowników była zwiększana stopniowo, po 5 użytkowników na sekundę, aż do osiągnięcia docelowej liczby użytkowników. Rezultaty testów były zapisywane po dziesięciu minutach od rozpoczęcia testu. 

Wyniki testów wydajnościowych zostały przedstawione w tabeli \ref{table:performance}. Śledzonymi metrykami były liczba żądań na sekundę, średnie czasy odpowiedzi oraz opóźnienie obsługi zdarzeń. Opóźnienie obsługi zdarzeń jest to czas, który upłynął od momentu wygenerowania zdarzenia przez system, do momentu jego obsłużenia przez wszystkie komponenty systemu.

\begin{longtable}{| m{0.25\linewidth} | m{0.2\linewidth} | m{0.25\linewidth} | m{0.2\linewidth} |}
    \caption{Rezultaty testów wydajnościowych}
    \label{table:performance} \\

    \hline
    Liczba użytkowników & Liczba żądań na sekundę & Czasy odpowiedzi & Opóźnienie obsługi zdarzeń \\ \hline\hline \endfirsthead \endfoot
    \hline \endlastfoot

    100 & 90,5 & Percentyl 50\% - 13 ms \newline Percentyl 95\% - 24 ms & 3 ms \\ \hline
    200 & 183,2 & Percentyl 50\% - 15 ms \newline Percentyl 95\% - 29 ms & 10 ms \\ \hline
    300 & 264,7 & Percentyl 50\% - 16 ms \newline Percentyl 95\% - 30 ms & 6 ms \\ \hline
    500 & 438,1 & Percentyl 50\% - 59 ms \newline Percentyl 95\% - 210 ms & 8,86 s \\ \hline
    1000 & 886,2 & Percentyl 50\% - 62 ms \newline Percentyl 95\% - 160 ms & 16,12 s \\ \hline
\end{longtable}

System zachowywał się stabilnie do momentu osiągnięcia liczby 500 użytkowników. W tym momencie, opóźnienie obsługi zdarzeń zaczęło gwałtownie rosnąć, co oznacza, że system nie był w stanie obsłużyć wszystkich zdarzeń w czasie rzeczywistym. Jednocześnie, nadal obsługiwał on wszystkie żądania w akceptowalnym czasie. Problem z opóźnieniem obsługi zdarzeń rósł wraz ze wzrostem liczby użytkowników. W przypadku 1000 użytkowników, opóźnienie obsługi zdarzeń wyniosło 16,12 sekundy.


Na rysunku \ref{fig:locust} przedstawiono przykładowy rezultat wykonania testów wydajnościowych. Na wykresie przedstawiono liczbę użytkowników, którzy wykonują akcje w systemie, liczbę zapytań na sekundę, które są wykonywane oraz średnie czasy odpowiedzi.

\begin{figure}[!h]
    \centering \includegraphics[width=1.0\linewidth]{locust.png}
    \caption{Przykładowy rezultat wykonania testów wydajnościowych}
    \label{fig:locust}
\end{figure}
\clearpage % Rozdziały zaczynamy od nowej strony.

\section{Podsumowanie}

\subsection{Wnioski}

\subsection{Stopień realizacji założeń}

\subsection{Możliwości dalszego rozwoju}

%---------------
% Bibliografia
%---------------
\cleardoublepage % Zaczynamy od nieparzystej strony
\printbibliography
\clearpage

% Wykaz symboli i skrótów.
% Pamiętaj, żeby posortować symbole alfabetycznie
% we własnym zakresie. Makro \acronymlist
% generuje właściwy tytuł sekcji, w zależności od języka.
% Makro \acronym dodaje skrót/symbol do listy,
% zapewniając podstawowe formatowanie.
\acronymlist
\acronym{EiTI}{Wydział Elektroniki i Technik Informacyjnych}
\acronym{PW}{Politechnika Warszawska}
\acronym{WEIRD}{ang. \emph{Western, Educated, Industrialized, Rich and Democratic}}
\vspace{0.8cm}

%--------------------------------------
% Spisy: rysunków, tabel, załączników
%--------------------------------------
\pagestyle{plain}

\listoffigurestoc    % Spis rysunków.
\vspace{1cm}         % vertical space
\listoftablestoc     % Spis tabel.
\vspace{1cm}         % vertical space
\listofappendicestoc % Spis załączników
\vspace{1cm}
\listoflistingstoc   % Spis listingów

% TODO spis listingów

%-------------
% Załączniki
%-------------

% % Obrazki i tabele w załącznikach nie trafiają do spisów
% \captionsetup[figure]{list=no}
% \captionsetup[table]{list=no}

% % Załącznik 1
% \clearpage
% \appendix{Nazwa załącznika 1}
% \lipsum[1-3]
% \begin{figure}[!h]
% 	\centering \includegraphics[width=0.5\linewidth]{logopw2.png}
% 	\caption{Obrazek w załączniku.}
% \end{figure}
% \lipsum[4-7]

% % Załącznik 2
% \clearpage
% \appendix{Nazwa załącznika 2}
% \lipsum[1-2]
% \begin{table}[!h] \centering
%     \caption{Tabela w załączniku.}
%     \begin{tabular} {| c | c | r |} \hline
%         Kolumna 1       & Kolumna 2 & Liczba \\ \hline\hline
%         cell1           & cell2     & 60     \\ \hline
%         \multicolumn{2}{|r|}{Suma:} & 123,45 \\ \hline
%     \end{tabular}
% \end{table}
% \lipsum[3-4]

% Używając powyższych spisów jako szablonu,
% możesz dodać również swój własny wykaz,
% np. spis algorytmów.

\end{document} % Dobranoc.
