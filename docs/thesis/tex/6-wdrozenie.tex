\clearpage % Rozdziały zaczynamy od nowej strony.

\section{Wdrożenie}

Aby zapewnić sprawne działanie systemu mikroserwisowego, należy zadbać o odpowiednie środowisko uruchomieniowe, w którym będą działać poszczególne mikroserwisy. Ponadto, należy zadbać o odpowiednie narzędzia, które pozwolą na automatyzację procesu wdrażania nowych wersji oprogramowania.

\subsection{CI/CD}

Aby usprawnić proces tworzenia i wdrażania oprogramowania, należy zadbać o odpowiednie narzędzia, które pozwolą na automatyzację tych procesów. Jednym z takich narzędzi jest CI/CD, czyli ciągła integracja i ciągłe wdrażanie. CI/CD pozwala na automatyzację procesu budowania aplikacji, testowania jej, a następnie wdrażania do środowiska produkcyjnego.

W ramach pracy inżynierskiej, został zaimplementowany potok CI/CD w ramach platformy GitHub Actions. Przykładowo, dla serwisu serwerowego potok ten składa się z następujących kroków:

\begin{enumerate}
    \item Przygotowanie środowiska z zainstalowanymi JDK 21 i Gradle \cite{gradle} 8.1.1
    \item Przeprowadzenie analizy statycznej kodu za pomocą narzędzia Detekt \cite{detekt}
    \item Przeprowadzenie testów jednostkowych i integracyjnych (komenda \texttt{./gradlew test})
    \item Zbudowanie aplikacji za pomocą Gradle (komenda \texttt{./gradlew build})
    \item Zbudowanie obrazu Docker za pomocą Docker Buildx (komenda \texttt{docker buildx build})
    \item Opublikowanie zbudowanego obrazu w GitHub Container Registry \cite{ghcr} (komenda \texttt{docker push})
    \item (opcjonalnie) Wdrożenie zbudowanego obrazu na klastrze Kubernetes (komenda \texttt{kubectl apply})
\end{enumerate}

Wszystkie komponenty systemu są budowane równolegle, co pozwala na skrócenie czasu budowania całego systemu. W przypadku, gdy któryś z kroków zakończy się niepowodzeniem, proces budowania zostaje przerwany, a programista otrzymują informację o błędzie.

W przypadku implementowanego systemu, cały potok wraz z wdrożeniem trwa około 3 minuty.

\subsection{Repozytorium artefaktów}

Aby zapewnić możliwość uruchomienia dowolnego elementu systemu, należy zadbać o dostępność wymaganych artefaktów (np. biblioteki \textit{shared}). W tym celu wykorzystano GitHub Actions Artifacts \cite{gaa}, które umożliwiają przechowywanie artefaktów np. systemu budowania Gradle i są zintegrowane z potokiem CI/CD.

\subsection{Konteneryzacja}

Wdrażanie systemów mikroserwisowych z wykorzystaniem Docker i Kubernetes stało się standardem w rozwoju nowoczesnych aplikacji webowych. Docker umożliwia tworzenie lekkich, przenośnych kontenerów dla każdego mikroserwisu, zapewniając izolację i spójność środowiska. Kontenery Docker mogą być następnie zarządzane i orkiestrowane za pomocą Kubernetes, platformy umożliwiającej automatyczne skalowanie, podział ruchu  i samonaprawę. Kubernetes zapewnia także mechanizmy do zarządzania siecią i komunikacji między kontenerami, co jest kluczowe dla sprawnego działania rozproszonych systemów mikroserwisowych.

TODO Dockerfile

\subsection{Orkiestracja kontenerów}

TODO k8s yaml

\subsection{Skalowanie}

\subsection{Chmura Google Cloud Platform}

Serwisy te zostały wdrożone jako osobne kontenery, orkiestrowane za pomocą narzędzia Kubernetes, na prywatnym klastrze uruchomionym w chmurze publicznej od firmy Google - Google Cloud Platform.

Klaster ten złożony jest z trzech maszyn typu e2-standard-4 (4 vCPU, 2 rdzenie, 16GB pamięci RAM, 100GB dysku twardego).

Domyślnie uruchomiono po jednej instancji serwisów obsługujących komendy (Dostawy, Zamówienia, Płatności, Restauracje) oraz po cztery instancje serwisów obsługujących większy ruch niż pozostałe (Zapytania oraz Sagi).

\subsection{Monitorowanie}

logi, axon console, axon server dashboard, prometheus, grafana

TODO screenshot grafany czy coś