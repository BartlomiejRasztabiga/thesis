\clearpage % Rozdziały zaczynamy od nowej strony.

\section{Wdrożenie}

\subsection{CI/CD}

\subsection{Repozytorium artefaktów}

\subsection{Konteneryzacja}

Wdrażanie systemów mikroserwisowych z wykorzystaniem Docker i Kubernetes stało się standardem w rozwoju nowoczesnych aplikacji webowych. Docker umożliwia tworzenie lekkich, przenośnych kontenerów dla każdego mikroserwisu, zapewniając izolację i spójność środowiska. Kontenery Docker mogą być następnie zarządzane i orkiestrowane za pomocą Kubernetes, platformy umożliwiającej automatyczne skalowanie, podział ruchu  i samonaprawę. Kubernetes zapewnia także mechanizmy do zarządzania siecią i komunikacji między kontenerami, co jest kluczowe dla sprawnego działania rozproszonych systemów mikroserwisowych.

\subsection{Orkiestracja kontenerów}

\subsection{Skalowanie}

\subsection{Chmura Google Cloud Platform}

Serwisy te zostały wdrożone jako osobne kontenery, orkiestrowane za pomocą narzędzia Kubernetes, na prywatnym klastrze uruchomionym w chmurze publicznej od firmy Google - Google Cloud Platform.

Klaster ten złożony jest z trzech maszyn typu e2-standard-4 (4 vCPU, 2 rdzenie, 16GB pamięci RAM, 100GB dysku twardego).

Domyślnie uruchomiono po jednej instancji serwisów obsługujących komendy (Dostawy, Zamówienia, Płatności, Restauracje) oraz po cztery instancje serwisów obsługujących większy ruch niż pozostałe (Zapytania oraz Sagi).

\subsection{Monitorowanie}

logi, axon console, axon server dashboard, prometheus, grafana