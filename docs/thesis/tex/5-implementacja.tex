\clearpage % Rozdziały zaczynamy od nowej strony.

\section{Implementacja}

Rozdział ten ma na celu szczegółowy opis implementacji systemu. Zostaną w nim przedstawione wykorzystane narzędzia, podział na moduły oraz szczegóły implementacyjne. Swoje miejsce znajdą tutaj również fragmenty kodu źródłowego, które mają na celu ułatwić zrozumienie sposobu działania systemu.

\subsection{Część serwerowa}

Część serwerową systemu stanowi aplikacja o architekturze mikroserwisowej, napisana w języku Kotlin z wykorzystaniem frameworków Spring Boot oraz Axon Framework.

\subsubsection{Użyte narzędzia}

W części serwerowej aplikacji wykorzystano następujące najważniejsze narzędzia:

\textbf{Kotlin} \cite{kotlin} to język programowania stworzony przez firmę JetBrains, działający na maszynie wirtualnej Javy (JVM) \cite{jvm}. Kotlin jest językiem statycznie typowanym, który łączy w sobie cechy zarówno języków obiektowych, jak i funkcyjnych. Jest on kompilowany do kodu bajtowego Javy, a jego składnia jest w dużej mierze zgodna z Javą, co czyni go łatwym do nauki i zrozumienia dla programistów Javy. Kotlin jest językiem wieloplatformowym, co oznacza, że może być kompilowany do kodu bajtowego Javy, kodu bajtowego Javy na Androida, kodu JavaScript oraz kodu natywnego. Kotlin jest językiem ogólnego przeznaczenia, który może być wykorzystywany do tworzenia aplikacji webowych, mobilnych, desktopowych, a nawet do tworzenia skryptów. Jego jedną z ważniejszych zalet jest wprowadzenie nullowalności na poziomie systemu typów, co pozwala na wykrywanie błędów związanych z niepoprawnym użyciem wartości null w czasie kompilacji, a nie w czasie działania programu.

\textbf{Spring Boot} \cite{springboot} to framework w ekosystemie Spring, który ułatwia tworzenie i rozwijanie aplikacji webowych oraz mikroserwisów w językach Java i Kotlin. Spring Boot oferuje wsparcie dla mikroserwisów, ułatwiając ich tworzenie, testowanie i wdrażanie, co czyni go popularnym wyborem wśród programistów pracujących nad nowoczesnymi, skalowalnymi aplikacjami.

Spring Boot ma silne powiązanie z koncepcją "cloud native", która odnosi się do projektowania aplikacji specjalnie na potrzeby chmury. Spring Boot ułatwia tworzenie aplikacji mikroserwisowych, które są nieodzownym elementem architektury cloud native, dostarczając funkcjonalności takie jak łatwa integracja z kontenerami (np. Docker), obsługa konfiguracji zewnętrznej, zarządzanie usługami przez service discovery oraz wspieranie wzorców takich jak circuit breaker. Te cechy sprawiają, że Spring Boot jest idealnym wyborem dla tworzenia aplikacji przygotowanych do działania w środowiskach chmurowych, oferujących skalowalność, elastyczność i odporność.

\textbf{Axon Framework} to framework do tworzenia aplikacji w architekturze opartej na zdarzeniach (event-driven) i wzorcu CQRS. Wspiera również DDD, Event Sourcing oraz wzorzec Saga. Axon Framework jest napisany w języku Java, ale może być wykorzystywany również w języku Kotlin. Axon Framework dostarcza abstrakcje do tworzenia aplikacji opartych na zdarzeniach, takich jak agregaty, komendy, zdarzenia, szyna wiadomości, sagi, itp. Axon Framework jest narzędziem open source, rozwijanym przez firmę AxonIQ.

\textbf{Axon Server} jest infrastrukturalnym elementem ekosystemu Axon, odgrywającym rolę kolejki komunikatów oraz magazynu zdarzeń. Oferuje on również narzędzia do monitoringu i zarządzania aplikacjami opartymi na Axon Framework.

\subsubsection{Architektura heksagonalna}

Część serwerowa aplikacji została zaimplementowana zgodnie z architekturą heksagonalną (ang. hexagonal architecture), która jest jedną z popularnych architektur aplikacji serwerowych, wykorzystywanych przy złożonych projektach informatycznych. Architektura ta jest również znana pod nazwą architektury czystej (ang. clean architecture) lub architektury portów i adapterów (ang. ports and adapters architecture). Została ona zaproponowana przez Alistaira Cockburna w 2005 roku \cite{cockburn2005hexagonal}.

Architektura heksagonalna jest architekturą warstwową, która składa się z trzech warstw: warstwy adapterów, warstwy dziedziny oraz warstwy infrastruktury. Warstwa dziedziny jest główną warstwą aplikacji, która zawiera logikę biznesową. Warstwa adapterów jest warstwą zewnętrzną, która zawiera adaptery wejściowe i wyjściowe, które są odpowiedzialne za komunikację z zewnętrznymi systemami. Warstwa infrastruktury jest warstwą wewnętrzną, która zawiera implementację adapterów wejściowych i wyjściowych. Warstwa infrastruktury jest odpowiedzialna za konfigurację aplikacji oraz za integrację z zewnętrznymi systemami (np. bazą danych, systemem plików itp.).

\subsubsection{Podział na pakiety}

Z powodu zastosowania architektury heksagonalnej, część serwerowa aplikacji została podzielona na pakiety zgodnie z warstwami tejże architektury. Podział na pakiety przykładowego serwisu został przedstawiony na wycinku \ref{lst:server-packages}.

\begin{lstlisting}[caption={Podział na pakiety części serwerowej projektu},label={lst:server-packages},captionpos=b]
- adapter/
    - in/
    - out/
- domain/
    - query/
    - command/
- config/
- infrastructure/
- Application.kt
\end{lstlisting}

W kolejnych podrozdziałach zostaną przedstawione szczegóły implementacyjne poszczególnych warstw wraz z ich opisem.

\subsubsection{Warstwa adapterów wejściowych} 

Warstwa ta obejmuje pakiet \textit{adapter.in}. Zawiera ona tzw. adaptery wejściowe, czyli komponenty odpowiedzialne za rozpoczynanie przepływu sterowania w aplikacji. Adaptery wejściowe są odpowiedzialne za obsługę zapytań i komend, które są wysyłane do aplikacji przez użytkowników lub inne systemy. W implementowanym systemie adapterami wejściowymi są kontrolery REST API oraz serwisy nasłuchujące na zdarzenia z kolejki komunikatów.

Na wycinku \ref{lst:server-in-adapter} przedstawiono przykładowy kod kontrolera REST API, który obsługuje zapytanie HTTP POST na ścieżce \textit{/api/v1/orders}. Zapytanie to rozpoczyna zamówienie ze strony użytkownika zamawiającego. W ciele zapytania znajduje się obiekt JSON, który jest mapowany na obiekt klasy \textit{StartOrderRequest}. Obiekt ten jest następnie przekazywany do komponentu \textit{ReactorCommandGateway}, który jest odpowiedzialny za wysłanie komendy \textit{StartOrderCommand} do szyny komend. Komenda ta jest następnie przekazywana do odpowiedniego agregatu dziedzinowego, który jest odpowiedzialny za jej obsługę. Rezultatem synchronicznym obsługi komendy jest nadany identyfikator zamówienia, w postacu UUID, który jest zwracany w odpowiedzi HTTP. 

\begin{lstlisting}[caption={Kod kontrolera REST API obsługującego zapytania dotyczące zamówień},label={lst:server-in-adapter},captionpos=b,language=Kotlin,numbers=left]
@RestController
@RequestMapping("/api/v1/orders")
class OrderController(
    private val reactorCommandGateway: ReactorCommandGateway
) {
    @PostMapping
    @ResponseStatus(HttpStatus.CREATED)
    @PreAuthorize("hasAnyAuthority('${Scopes.ORDER.WRITE}')")
    fun startOrder(
        @RequestBody request: StartOrderRequest,
        exchange: ServerWebExchange
    ): Mono<UuidWrapper> {
        val command = mapToStartOrderCommand(request, exchange)
        val id = reactorCommandGateway.send<UUID>(command)
        return id.map { UuidWrapper(it) }
    }
}
\end{lstlisting}

Kod kontrolera został napisany z wykorzystaniem biblioteki programowania reaktywnego Spring WebFlux, która pozwala na obsługę żądań HTTP w sposób asynchroniczny. Pozwala to na obsługę większej liczby żądań przy użyciu mniejszej liczby wątków, co przekłada się na większą wydajność aplikacji.

Poza kodem obsługującym żądanie HTTP, kontroler zawiera również adnotacje, które definiują uprawnienia wymagane do wykonania żądania. W tym przypadku żądanie wymaga posiadania uprawnień \textit{ORDER.WRITE}, które są zdefiniowane w klasie \textit{Scopes}. Jest to część biblioteki Spring Security, implementującej standard OAuth 2.0.

\subsubsection{Warstwa adapterów wyjściowych} 

Warstwa ta obejmuje pakiet \textit{adapter.out}. Zawiera ona tzw. adaptery wyjściowe, czyli komponenty odpowiedzialne za komunikację z zewnętrznymi systemami. W implementowanym systemie adapterami wyjściowymi są komponenty oferujące dostęp do bazy danych oraz komunikację z zewnętrznymi systemami np. Google Maps Platform.

Na wycinku \ref{lst:server-out-adapter} przedstawiono przykładowy kod komponentu oferującego dostęp do bazy danych. Jest to implementacja (adapter) interfejsu (portu) \textit{OrderDeliveryRepository}, który jest odpowiedzialny za zapisywanie i odczytywanie encji \textit{OrderDeliveryEntity} z bazy danych. Implementacja ta wykorzystuje repozytorium Spring Data, które jest komponentem oferującym dostęp do bazy MongoDB. Repozytorium to jest wstrzykiwane do klasy \textit{DbOrderDeliveryRepository} przy pomocy mechanizmu wstrzykiwania zależności oferowanego przez framework Spring. 

\begin{lstlisting}[caption={Kod implementacji repozytorium dziedzinowego projekcji Zamówienia},label={lst:server-out-adapter},captionpos=b,language=Kotlin,numbers=left]
@Service
class DbOrderDeliveryRepository(
    private val repository: SpringDataOrderDeliveryRepository
) : OrderDeliveryRepository {
    override fun save(delivery: OrderDeliveryEntity) {
        repository.save(delivery).block()
    }

    override fun load(id: UUID): OrderDeliveryEntity? {
        return repository.findById(id).block()
    }

    override fun loadOffers(): List<OrderDeliveryEntity> {
        return repository
            .findAllByStatus(DeliveryStatus.OFFER)
            .collectList().block() ?: listOf()
    }
}
\end{lstlisting}

\subsubsection{Warstwa dziedziny} 

Warstwa ta obejmuje pakiet \textit{domain}. Zawiera ona komponenty, które są odpowiedzialne za logikę biznesową aplikacji. W implementowanym systemie komponentami tymi są agregaty dziedzinowe, komendy, zdarzenia, porty, sagi, encje bazodanowe, klasy wyjątków, zapytania, komponenty budujące projekcje oraz obiekty transferu danych.

Warstwa dziedziny jest podzielona zgodnie ze wzorcem CQRS na część komend oraz część zapytań. Część komend jest zawarta w podpakiecie \textit{command}, a część zapytań w podpakiecie \textit{query}.

Na wycinku \ref{lst:server-domain} przedstawiono częściowy kod komponentów warstwy dziedziny odpowiadającej za rozpoczynanie zamówienia z perspektywy użytkownika systemu. Są to: agregat dziedzinowy \textit{Order}, komenda \textit{StartOrderCommand} oraz zdarzenie \textit{OrderStartedEvent}. Przedstawiono również zawartość portu \textit{OrderVerificationPort}, który jest odpowiedzialny za wstępną walidację zamówienia.

\begin{lstlisting}[caption={Kod komponentów odpowiedzialnych za rozpoczynanie zamówień},label={lst:server-domain},captionpos=b,language=Kotlin,numbers=left]
@Aggregate
internal class Order {

    @AggregateIdentifier
    private lateinit var id: UUID
    private lateinit var status: OrderStatus

    @AggregateMember
    private val items: MutableMap<UUID, Int> = mutableMapOf()

    @CommandHandler
    constructor(
        command: StartOrderCommand,
        orderVerificationPort: OrderVerificationPort
    ) {
        require(
            orderVerificationPort.restaurantExists(command.restaurantId)
        )
        require(
            orderVerificationPort.isRestaurantOpen(command.restaurantId)
        )

        apply(
            OrderStartedEvent(
                orderId = command.orderId,
                restaurantId = command.restaurantId,
                userId = command.userId,
                status = OrderStatus.CREATED
            )
        )
    }
}

data class StartOrderCommand(
    @TargetAggregateIdentifier val orderId: UUID,
    val userId: String,
    val restaurantId: UUID
)

data class OrderStartedEvent(
    val orderId: UUID,
    val restaurantId: UUID,
    val userId: String,
    val status: OrderStatus
)

interface OrderVerificationPort {
    fun restaurantExists(restaurantId: UUID): Boolean
    fun isRestaurantOpen(restaurantId: UUID): Boolean
}
\end{lstlisting}

Kod ten jest wywoływany w momencie otrzymania komendy \textit{StartOrderCommand} przez agregat \textit{Order}. Komenda ta jest wysyłana do agregatu przez adapter wejściowy, zaprezentowany w rozdziale 5.1.4. Agregat \textit{Order} jest odpowiedzialny za wstępną walidację zamówienia, która jest realizowana przy pomocy portu \textit{OrderVerificationPort}. W przypadku tej komendy system weryfikuje czy podana restauracja istnieje i czy jest otwarta. Port walidacyjny jest interfejsem, implementowanym przez odpowiedni adapter wyjściowy.

W przypadku, gdy wstępna walidacja zamówienia zakończy się sukcesem, agregat \textit{Order} emituje zdarzenie \textit{OrderStartedEvent}, które jest zapisywane w magazynie zdarzeń, kontrolowanym przez Axon Server. Na zdarzenie nasłuchują inne komponenty systemu, które mogą na jego podstawie budować projekcje, wysyłać powiadomienia do użytkowników, lub podejmować inne akcje.

Dzięki zastosowaniu Axon Framework nie ma konieczności pisania kodu odpowiedzialnego za ładowanie agregatów dziedzinowych, wywoływanie odpowiednich metod obsługujących komendy czy publikowanie zdarzeń. Większość interakcji odbywa się bez udziału programisty, przy pomocy odpowiednich adnotacji, takich jak np. \textit{@Aggregate}, \textit{@AggregateIdentifier}, \textit{@CommandHandler} itp.

\subsubsection{Warstwa prezentacji/zapytań}

Warstwa ta obejmuje pakiet \textit{query}. Zawiera ona komponenty, które są odpowiedzialne za obsługę zapytań użytkowników systemu. W implementowanym systemie komponentami tymi są procesory zdarzeń budujące projekcje, zapytania oraz klasy pomocnicze służące do persystencji danych w bazie danych.

Na wycinku \ref{lst:server-domain-query} przedstawiono kod procesora zdarzeń budującego projekcję zamówienia. Procesor ten nasłuchuje na zdarzenie \textit{OrderStartedEvent} i na jego podstawie buduje projekcję zamówienia, która jest zapisywana w bazie danych. Procesor ten jest wywoływany automatycznie przez Axon Framework, w momencie otrzymania odpowiedniego zdarzenia.

\begin{lstlisting}[caption={Kod procesora zdarzeń budującego projekcję zamówienia},label={lst:server-domain-query},captionpos=b,language=Kotlin,numbers=left,showstringspaces=false]
@ProcessingGroup("projection")
class OrderHandler(
    private val orderRepository: OrderRepository,
    private val restaurantRepository: RestaurantRepository
) {

    @EventHandler
    fun on(event: OrderStartedEvent) {
        val restaurant = restaurantRepository.load(
            event.restaurantId
        )

        val restaurantLocation = restaurant?.location 
            ?: error("Restaurant not found")

        val entity = OrderEntity(
            id = event.orderId,
            restaurantId = event.restaurantId,
            restaurantLocation = restaurantLocation,
            userId = event.userId,
            status = OrderStatus.CREATED,
            items = mutableMapOf(),
            createdAt = Instant.now()
        )
        orderRepository.save(entity)
    }
}
\end{lstlisting}

\subsubsection{Warstwa konfiguracji}

Warstwa ta obejmuje pakiet \textit{config}. Zawiera ona komponenty, które są odpowiedzialne za konfigurację aplikacji.

Przykładowym komponentem w implementowanym systemie jest filtr zapytań, odpowiedzialny za odczytywanie atrybutu \textit{subject} z tokenu JWT oraz zapisywanie go w kontekście żądania. Jest to używane m.in. do poprawnej autoryzacji użytkownika w momencie wywoływania komendy (np. dostęp do zasobu ma tylko jego autor). Kod tego komponentu został przedstawiony na wycinku \ref{lst:server-config}.

\begin{lstlisting}[caption={Kod filtra zapytań},label={lst:server-config},captionpos=b,language=Kotlin,numbers=left,showstringspaces=false]
@Component
@Profile("!nosecurity")
class UserContextWebFilter : WebFilter {
    override fun filter(
        exchange: ServerWebExchange,
        chain: WebFilterChain
    ): Mono<Void> {
        return ReactiveSecurityContextHolder.getContext()
            .mapNotNull { it.authentication.principal }
            .cast(Jwt::class.java)
            .doOnNext {
                val userId = it.subject.removePrefix("auth0|")
                exchange.setUserId(userId)
            }
            .then(chain.filter(exchange))
    }
}
\end{lstlisting}

Dzięki zastosowaniu adnotacji \textit{@Profile("!nosecurity")} możliwe jest wyłączenie tego filtra w trybie \textit{nosecurity}, który jest używany w testach wydajnościowych, w których wyłączone są mechanizmy uwierzytelniania i autoryzacji.

\subsubsection{Warstwa infrastruktury}

Warstwa ta obejmuje pakiet \textit{infrastructure}. Zawiera ona kod, niezwiązany z domeną aplikacji, który jest odpowiedzialny za integrację z konkretnymi technologiami, np. bazą danych, systemem plików itp. W implementowanym systemie komponentami tej warstwy są np. implementacje repozytoriów bazodanowych używające konkretnej technologii - Spring Data MongoDB. Są to również komponenty służące do komunikacji z zewnętrznymi usługami, np. klient Google Maps Platform.

Na wycinku \ref{lst:server-infrastructure} przedstawiono przykładowy kod implementacji repozytorium bazodanowego. Jest to implementacja interfejsu \textit{ReactiveMongoRepository}, który jest komponentem oferującym dostęp do bazy MongoDB. Mimo faktu, że repozytorium to jest interfejsem, a nie klasą, to dzięki mechanizmowi refleksji oferowanemu przez framework Spring, implementacja ta jest automatycznie generowana w czasie uruchamiania aplikacji.

\begin{lstlisting}[caption={Kod implementacji repozytorium bazodanowego},label={lst:server-infrastructure},captionpos=b,language=Kotlin,numbers=left,showstringspaces=false]
@Repository
interface SpringDataOrderRepository 
    : ReactiveMongoRepository<OrderEntity, UUID> {

    fun findAllByUserId(userId: String): Flux<OrderEntity>
}
\end{lstlisting}

Repozytorium bazowe \textit{ReactiveMongoRepository} zawiera podstawowe metody do operacji na bazie danych, takie jak np. \textit{save}, \textit{findById}, \textit{findAll} itp. Repozytorium \textit{SpringDataOrderRepository} rozszerza repozytorium bazowe i dodaje do niego dodatkową metodę \textit{findAllByUserId}, która zwraca wszystkie zamówienia danego użytkownika.

\subsubsection{Współdzielony kod}

Kod współdzielony przez różne serwisy w ramach systemu został umieszczony w osobnym module \textit{shared}. Zawiera on m.in. klasy reprezentujące obiekty transferu danych, wyjątki, stałe, wspólną konfigurację itp.

Moduł ten jest wersjonowany i publikowany jako artefakt systemu budowania Maven, dzięki czemu może być umieszczony w centralnym repozytorium i importowany przez inne projekty.

Przykład współdzielonej klasy zaprezentowany jest na wycinku \ref{lst:server-shared}

\begin{lstlisting}[caption={Kod klasy konfigurującej mechanizm uwierzytelniania i autoryzacji},label={lst:server-shared},captionpos=b,language=Kotlin,numbers=left,showstringspaces=false]
@Configuration
@EnableReactiveMethodSecurity
@EnableWebFluxSecurity
@Profile("!nosecurity")
class SecurityConfig {

    @Bean
    fun springSecurityFilterChain(
        http: ServerHttpSecurity
    ): SecurityWebFilterChain {
        return http.authorizeExchange {
            it.pathMatchers("/actuator/**").permitAll()
            it.pathMatchers("/api/v1/payments/stripe-webhook").permitAll()
            it.anyExchange().authenticated()
        }.csrf {
            it.disable()
        }.cors {
            it.disable()
        }.oauth2ResourceServer {
            it.jwt {}
        }.build()
    }
}
\end{lstlisting}

Klasa ta jest odpowiedzialna za konfigurację mechanizmu uwierzytelniania i autoryzacji. Została ona umieszczona w module \textit{shared}, ponieważ jest ona wspólna dla wszystkich serwisów w ramach systemu. Uruchamia ona mechanizm uwierzytelniania i autoryzacji oferowany przez framework Spring Security. Uwierzytelnianie odbywa się przy pomocy tokenów JWT. Dodatkowo wyłącza ona mechanizmy obrony przed atakiem CSRF oraz CORS, które są niepotrzebne w przypadku zastosowanej infrastruktury sieciowej systemu.

Domyślnie, żądania na wszystkie ścieżki aplikacji wymagają uwierzytelniania. Są jednak dwie ścieżki, gdzie takie mechanizmy są wyłączone. Pierwsza z nich to ścieżka \textit{/actuator/**}, która jest wykorzystywana przez framework Spring Boot m.in. do publikowania statusu aplikacji, wykorzystywanego przez orkiestratora kontenerów Kubernetes. Druga ścieżka to \textit{/api/v1/payments/stripe-webhook}, która jest wykorzystywana przez serwis odpowiedzialny za płatności, do obsługi webhooków z serwisu Stripe.

\subsection{Część kliencka}

Część kliencka aplikacji została zaimplementowana jako aplikacja przeglądarkowa typu Multi Page Application (MPA) w języku TypeScript z wykorzystaniem frameworków React oraz Remix.

W przeciwieństwie do popularnego obecnie podejścia Single Page Application (SPA), w którym cała aplikacja jest ładowana przy pierwszym wejściu użytkownika na stronę, w podejściu MPA każda strona jest ładowana osobno. MPA wymaga przeładowywania całej strony podczas nawigacji między różnymi widokami, co jest wadą tego podejścia. Zaletą jest jednak to, że każda strona może być ładowana niezależnie, co pozwala na szybsze ładowanie aplikacji, a także na łatwiejsze indeksowanie przez wyszukiwarki internetowe. Upraszcza to również proces tworzenia aplikacji, ponieważ nie ma potrzeby implementowania zaawansowanym mechanizmów trasowania lub zarządzania stanem.

\subsubsection{Użyte narzędzia}

W części klienckiej aplikacji wykorzystano następujące najważniejsze narzędzia:

\textbf{TypeScript} \cite{typescript} to język programowania stworzony przez firmę Microsoft, będący rozszerzeniem języka JavaScript, poprzed wprowadzenie statycznego typowania, co pozwala programistom na wykrywanie błędów w czasie kompilacji i znacząco ułatwia zarządzanie dużymi projektami. 

\textbf{React} \cite{react} to biblioteka JavaScript, która jest wykorzystywana do tworzenia interfejsów użytkownika. React jest deklaratywną biblioteką open source, rozwijaną przez firmę Facebook, która pozwala na tworzenie interfejsów użytkownika przy pomocy komponentów. Może ona być wykorzystywana do tworzenia zarówno aplikacji webowych, jak i mobilnych. W przeciwieństwie do poprzednich podejść w tworzeniu aplikacji internetowych React stosuje technikę wirtualnego drzewa dokumentu (ang. virtual DOM) do renderowania interfejsu użytkownika. Technika ta polega na przechowywaniu w pamięci drzewa DOM, które jest reprezentacją interfejsu użytkownika. W momencie zmiany stanu aplikacji, DOM jest aktualizowany, a następnie porównywany z tym przechowywanym w pamięci. Na podstawie tego porównania, React aktualizuje tylko te elementy interfejsu użytkownika, które uległy zmianie. Dzięki temu renderowanie interfejsu użytkownika jest znacznie szybsze, co przekłada się na lepszą wydajność aplikacji.

\textbf{Remix} \cite{remix} to framework do tworzenia aplikacji internetowych typu MPA, oparty na modelu SSR (ang. server side rendering), czyli renderowania interfejsu użytkownika po stronie serwera. Pozwala on na pobieranie danych na poziomie trasowania, co oznacza, że dane są ładowane przed renderowaniem komponentów, co pozwala na szybsze ładowanie aplikacji i lepsze pozycjonowanie w wyszukiwarkach internetowych.

\subsubsection{Podział na katalogi}

Część kliencka aplikacji została podzielona na foldery, grupując w ten sposób różne rodzaje elementów systemu. Podział przedstawiono na wycinku \ref{lst:client-packages}.

\begin{lstlisting}[caption={Podział na foldery części klienckiej projektu},label={lst:client-packages},captionpos=b]
- components/
- routes/
- server/
- services/
- styles/
\end{lstlisting}

W kolejnych podrozdziałach zostaną przedstawione szczegóły implementacyjne poszczególnych warstw wraz z ich opisem.

\subsubsection{Komponenty interfejsu użytkownika}

Komponenty interfejsu użytkownika zostały umieszczone w folderze \textit{components}. Zostały one podzielone na foldery, grupujące komponenty używane w częściach aplikacji przeznaczonych dla różnych klas użytkowników (zamawiających, managerów restauracji i kurierów).

Przykładowy komponent został przedstawiony na wycinku \ref{lst:client-component}. Jest to komponent \textit{Topbar} odpowiedzialny za wyświetlanie paska nawigacji dla managerów restauracji. Przyjmuje on dwa parametry: \textit{payee} oraz \textit{restaurant}. Pierwszy z nich jest obiektem zawierającym informacje o aktualnie zalogowanym użytkowniku, a drugi zawiera informacje o restauracji, w której aktualnie pracuje użytkownik. Komponent ten wyświetla nazwę restauracji oraz jej aktualną dostępność. Jest ona wyświetlana w formie przycisku, który po kliknięciu wysyła żądanie HTTP PUT do serwera, w celu zmiany dostępności restauracji.

\begin{lstlisting}[caption={Kod komponentu interfejsu użytkownika - pasek nawigacji managera restauracji},label={lst:client-component},captionpos=b,language=JavaScript,numbers=left,showstringspaces=false]
export default function Topbar(props: TopbarProps) {
    return (
      <header
        className="..."
      >
        <span>Restaurant "{props.restaurant.name}"</span>
        <Form method="post">
          <span className="align-middle inline-block">
            {props.restaurant.availability}
          </span>
          <input
            type="hidden"
            name="availability"
            value={props.restaurant.availability}
          />
          <input
            type="hidden"
            name="restaurantId"
            value={props.restaurant.id}
          />
          <Button 
            type="submit" 
            name="_action" 
            value="update_availability"
          >
            CHANGE
          </Button>
        </Form>
      </header>
    );
}
  
export interface TopbarProps {
  payee: PayeeResponse;
  restaurant: RestaurantResponse;
}
\end{lstlisting}

Odpowiedzialność za dostarczenie danych oraz obsługę formularza komponent ten deleguje do swojego rodzica. Jest to podejście typowe dla biblioteki React, która promuje podejście kompozycyjne do tworzenia interfejsów użytkownika.

\subsubsection{Obsługa ścieżek}

Komponenty aplikacji odpowiedzialne za obsługę ścieżek zostały umieszczone w folderze \textit{routes}. W frameworku Remix komponenty te są punktami startowymi podróży użytkownika po aplikacji. Na przykład, po podaniu w przeglądarce użytkownika ścieżki \textit{/v2/courier/delivery} aplikacja uruchomi plik \textit{v2.courier.delivery.tsx}, który od tego momentu jest odpowiedzialny za ładowanie danych, obsługę formularzy i renderowanie komponentów.

Na wycinku \ref{lst:client-route} przedstawiono kod komponentu obsługującego ścieżkę \textit{/v2/ordering/restaurants} wyświetlającą listę restauracji, które może wybrać użytkownik zamawiający.

\begin{lstlisting}[caption={Kod ścieżki wyświetlającej listę dostępnych restauracji - \textit{/v2/ordering/restaurants}},label={lst:client-route},captionpos=b,language=JavaScript,numbers=left,showstringspaces=false]
export async function loader({ request }: LoaderArgs) {
    const currentUser = await getCurrentUser(request);
    const restaurants = await getRestaurants(request);
  
    const activeOrderId = await getOrderId(request);
    const activeOrder = await getOrder(request, activeOrderId)
  
    return json({ restaurants, currentUser, activeOrder });
}

export async function action({ request, params }: ActionArgs) {
    const formData = await request.formData();
    ...
}

export default function V2RestaurantsPage() {
  const data = useLoaderData<typeof loader>();
  const navigate = useNavigate();

  const openRestaurants = data.restaurants.filter(
    (restaurant) => restaurant.availability === "OPEN",
  );

  return (
    <div className="flex flex-col h-full overflow-x-hidden">
      <div>
        <Topbar user={data.currentUser} />
      </div>
      <div className="h-full">
        <div className="flex flex-col w-80 mx-auto">
          {openRestaurants.map((restaurant, key) => (
            <Card key={key} className={"my-4"}>
              <CardActionArea>
                <NavLink to={restaurant.id}>
                  <CardMedia
                    component="img"
                    image={restaurant.imageUrl}
                  />
                  <CardContent>
                    ...
                  </CardContent>
                </NavLink>
              </CardActionArea>
            </Card>
          ))}
        </div>
      </div>
      <div>
        <BottomNavbar />
      </div>
    </div>
  );
}      
\end{lstlisting}
      
Charakterystycznymi dla aplikacji napisanych z użyciem frameworka Remix, jest obecność dwóch funkcji w każdym komponencie obsługującym ścieżkę. 

Pierwszą z nich jest funkcja \textit{loader}, która jest odpowiedzialna za ładowanie danych. Funkcja ta jest wywoływana po stronie serwera, w momencie wejścia użytkownika na daną ścieżkę. Dane zwracane przez tę funkcję są następnie przekazywane do funkcji, która jest odpowiedzialna za renderowanie komponentów. Funkcja ta jest wywoływana po stronie klienta, w momencie przejścia użytkownika na daną ścieżkę. Dzięki temu użytkownik nie musi czekać na załadowanie danych, aby zobaczyć interfejs użytkownika. 

Drugą funkcją jest \textit{action}, która jest odpowiedzialna za obsługę formularzy, wysyłanych przez użytkownika w celu dokonania zmiany danych w systemie. Funkcja ta jest wywoływana po stronie serwera, w momencie wysłania formularza przez użytkownika.

\subsubsection{Komunikacja z częścią serwerową}

Funkcje odpowiedzialne za wykonywanie żądań HTTP do części serwerowej zostały umieszczone w katalogu \textit{server}. Pliki te zawierają również obiekty transferu danych, które są używane do przesyłania danych pomiędzy częścią kliencką a serwerową.

Na wycinku \ref{lst:client-server} przedstawiono przykładowy kod funkcji wywołujących żądania HTTP do części serwerowej. Funkcje te są wywoływane przez funkcje \textit{loader} oraz \textit{action} opisane w poprzednim podrozdziale. Kod ten wykorzystuje bibliotekę \textit{axios}, która jest odpowiedzialna za wykonywanie żądań HTTP. Co ważne, funkcje te są wywoływane po stronie serwera, a nie klienta, co pozwala np. na ukrycie kluczy dostępowych do zewnętrznych usług, takich jak np. Google Maps Platform.

\begin{lstlisting}[caption={Kod funkcji wywołujących żadania HTTP do części serwerowej},label={lst:client-server},captionpos=b,language=JavaScript,numbers=left,showstringspaces=false]
export const getOrders = async (
    request: Request
): Promise<OrderResponse[]> => {
    const axios = await getAxios(request);
    return axios.get(`/api/v2/orders`)
      .then((res) => res.data);
};
  
export const startOrder = async (
    request: Request,
    restaurantId: string
): Promise<UuidWrapper> => {
    const axios = await getAxios(request);
    return axios.post(`/api/v1/orders`, { restaurantId })
      .then((res) => res.data);
};

export interface OrderResponse {
    id: string;
    restaurantId: string;
    restaurantLocation: Location;
    deliveryLocation: Location;
    courierLocation: Location;
    total: number;
    deliveryFee: number;
    userId: string;
    status: OrderStatus;
    items: OrderItem[];
    itemsTotal: number;
    paymentId: string;
    paymentSessionUrl: string;
    createdAt: string;
    events: OrderEvent[];
}
\end{lstlisting}

\subsubsection{Współdzielony kod} 

Kod współdzielony przez różne komponenty w ramach systemu został umieszczony w osobnym folderze \textit{services}. Zawiera on m.in. kod odpowiedzialny za manipulowanie sesją użytkownika w postaci ciasteczka HTTP. Jego fragment został zaprezentowany na listingu \ref{lst:client-service}.

\begin{lstlisting}[caption={Kod funkcji manipulujących sesją użytkownika},label={lst:client-service},captionpos=b,language=JavaScript,numbers=left,showstringspaces=false]
export const sessionStorage = createCookieSessionStorage({
    cookie: {
      name: "__session",
      httpOnly: true,
      path: "/",
      sameSite: "lax",
      secrets: [process.env.SESSION_SECRET],
      secure: process.env.NODE_ENV === "production",
    },
});

export async function getSession(request: Request) {
    const cookie = request.headers.get("Cookie");
    return sessionStorage.getSession(cookie);
}

export async function setOrderId(request: Request, orderId: string) {
    const session = await getSession(request);
    session.set("orderId", orderId);
    return sessionStorage.commitSession(session);
}

export async function getOrderId(request: Request) {
    const session = await getSession(request);
    return session.get("orderId");
}
\end{lstlisting}

\subsubsection{Przechowywanie stanu aplikacji}

Z powodu charakteru aplikacji typu MPA, w której każda strona jest ładowana osobno, nie ma potrzeby przechowywania stanu aplikacji po stronie klienta. Wszystkie dane są pobierane z serwera w momencie przejścia użytkownika na daną stronę. Cały stan aplikacji jest przechowywany po stronie serwera, w postaci sesji użytkownika, która jest przechowywana w postaci ciasteczka HTTP w przeglądarce klienta. Oznacza to, że w momencie zmiany przeglądarki lub urządzenia, użytkownik traci swój stan aplikacji. Ciasteczko to jest typu HTTP Only, co oznacza, że nie jest ono dostępne dla kodu JavaScript, co zwiększa bezpieczeństwo aplikacji. Jest ono dodatkowo szyfrowane przy pomocy tajnego klucza, który jest przechowywany po stronie serwera.