\clearpage % Rozdziały zaczynamy od nowej strony.

\section{Implementacja}

Rozdział ten ma na celu szczegółowy opis implementacji systemu. Zostaną w nim przedstawione wykorzystane narzędzia, podział na moduły oraz szczegóły implementacyjne. Swoje miejsce znajdą tutaj również fragmenty kodu źródłowego, które mają na celu ułatwić zrozumienie sposobu działania systemu.

\subsection{Część serwerowa}

Część serwerową systemu stanowi aplikacja o architekturze mikroserwisowej, napisana w języku Kotlin z wykorzystaniem frameworków Spring Boot oraz Axon Framework.

\subsubsection{Użyte narzędzia}

W części serwerowej aplikacji wykorzystano następujące najważniejsze narzędzia:

\textbf{Kotlin} \cite{kotlin} to język programowania stworzony przez firmę JetBrains, działający na maszynie wirtualnej Javy (JVM) \cite{jvm}. Kotlin jest językiem statycznie typowanym, który łączy w sobie cechy zarówno języków obiektowych, jak i funkcyjnych. Jest on kompilowany do kodu bajtowego Javy, a jego składnia jest w dużej mierze zgodna z Javą, co czyni go łatwym do nauki i zrozumienia dla programistów Javy. Kotlin jest językiem wieloplatformowym, co oznacza, że może być kompilowany do kodu bajtowego Javy, kodu bajtowego Javy na Androida, kodu JavaScript oraz kodu natywnego. Kotlin jest językiem ogólnego przeznaczenia, który może być wykorzystywany do tworzenia aplikacji webowych, mobilnych, desktopowych, a nawet do tworzenia skryptów. Jego jedną z ważniejszych zalet jest wprowadzenie nullowalności na poziomie systemu typów, co pozwala na wykrywanie błędów związanych z niepoprawnym użyciem wartości null w czasie kompilacji, a nie w czasie działania programu.

\textbf{Spring Boot} \cite{springboot} to framework w ekosystemie Spring, który ułatwia tworzenie i rozwijanie aplikacji webowych oraz mikroserwisów w językach Java i Kotlin. Spring Boot oferuje wsparcie dla mikroserwisów, ułatwiając ich tworzenie, testowanie i wdrażanie, co czyni go popularnym wyborem wśród programistów pracujących nad nowoczesnymi, skalowalnymi aplikacjami.

Spring Boot ma silne powiązanie z koncepcją "cloud native", która odnosi się do projektowania aplikacji specjalnie na potrzeby chmury. Spring Boot ułatwia tworzenie aplikacji mikroserwisowych, które są nieodzownym elementem architektury cloud native, dostarczając funkcjonalności takie jak łatwa integracja z kontenerami (np. Docker), obsługa konfiguracji zewnętrznej, zarządzanie usługami przez service discovery oraz wspieranie wzorców takich jak circuit breaker. Te cechy sprawiają, że Spring Boot jest idealnym wyborem dla tworzenia aplikacji przygotowanych do działania w środowiskach chmurowych, oferujących skalowalność, elastyczność i odporność.

\textbf{Axon Framework} to framework do tworzenia aplikacji w architekturze opartej na zdarzeniach (event-driven) i wzorcu CQRS. Wspiera również DDD, Event Sourcing oraz wzorzec Saga. Axon Framework jest napisany w języku Java, ale może być wykorzystywany również w języku Kotlin. Axon Framework dostarcza abstrakcje do tworzenia aplikacji opartych na zdarzeniach, takich jak agregaty, komendy, zdarzenia, szyna wiadomości, sagi, itp. Axon Framework jest narzędziem open source, rozwijanym przez firmę AxonIQ.

\textbf{Axon Server} jest infrastrukturalnym elementem ekosystemu Axon, pełniącym rolę kolejki komunikatów oraz magazynu zdarzeń. Oferuje on również narzędzia do monitoringu i zarządzania aplikacjami opartymi na Axon Framework.

\subsubsection{Architektura heksagonalna}

Część serwerowa aplikacji została zaimplementowana zgodnie z architekturą heksagonalną (ang. hexagonal architecture), która jest jedną z popularnych architektur aplikacji serwerowych, wykorzystywanych przy złożonych projektach informatycznych. Architektura heksagonalna jest również znana pod nazwą architektury czystej (ang. clean architecture) lub architektury portów i adapterów (ang. ports and adapters architecture). Architektura heksagonalna została zaproponowana przez Alistaira Cockburna w 2005 roku \cite{cockburn2005hexagonal}.

Architektura heksagonalna jest architekturą warstwową, która składa się z trzech warstw: warstwy adapterów, warstwy dziedziny oraz warstwy infrastruktury. Warstwa dziedziny jest główną warstwą aplikacji, która zawiera logikę biznesową. Warstwa adapterów jest warstwą zewnętrzną, która zawiera adaptery wejściowe i wyjściowe, które są odpowiedzialne za komunikację z zewnętrznymi systemami. Warstwa infrastruktury jest warstwą wewnętrzną, która zawiera implementację adapterów wejściowych i wyjściowych. Warstwa infrastruktury jest odpowiedzialna za konfigurację aplikacji oraz za integrację z zewnętrznymi systemami (np. bazą danych, systemem plików, itp.).

\subsubsection{Podział na pakiety}

Z powodu zastosowania architektury heksagonalnej, część serwerowa aplikacji została podzielona na pakiety zgodnie z warstwami architektury heksagonalnej. Podział na pakiety został przedstawiony na listingu \ref{lst:server-packages}.

TODO

TOOD czemu zly numer?
TODO czemu spis wycinkow kodu jest wciety z lewej?
\begin{lstlisting}[caption={Podział na pakiety części serwerowej aplikacji},label={lst:server-packages},captionpos=b]
teest
\end{lstlisting}


\subsubsection{Warstwa adapterów wejściowych} adapter/in

\subsubsection{Warstwa adapterów wyjściowych} adapter/out

\subsubsection{Warstwa dziedziny} domain/command

\subsubsection{Warstwa prezentacji/zapytań} domain/query

\subsubsection{Warstwa infrastruktury} infrastructure

\subsubsection{Warstwa konfiguracji} config

\subsubsection{Współdzielony kod} shared

\subsection{Część kliencka}

\subsubsection{Użyte narzędzia}

\subsubsection{Podział na katalogi}

\subsubsection{Komponenty interfejsu użytkownika}

\subsubsection{Komunikacja z częścią serwerową}

\subsubsection{Obsługa ścieżek}

\subsubsection{Współdzielony kod}

\subsubsection{Przechowywanie stanu aplikacji}