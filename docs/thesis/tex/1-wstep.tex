\clearpage % Rozdziały zaczynamy od nowej strony.
\section{Wstęp}

\subsection{Motywacja}

Wybór tematu pracy dyplomowej został podyktowany chęcią pogłębienia wiedzy oraz zaprezentowania technik tworzenia skalowalnych systemów mikroserwisowych na podstawie konkretnej aplikacji.

Wybraną przez Autora dziedziną do tego celu jest branża gastronomiczna. W ostatnich latach, wraz z rozwojem techniki, zaczęło powstawać wiele aplikacji służących do zamawiania jedzenia z restauracji. Aplikacje te cieszą się ogromną popularnością, więc ich twórcy muszą zapewnić odpowiednią skalowalność systemu, aby móc obsłużyć duży ruch. Jednym z rozwiązań tego problemu jest zastosowanie skalowalnych architektur, np. architektury mikroserwisowej.

Jednym z największych wyzwań stojących przed twórcami takich systemów jest zapewnienie niezawodności i ciągłości działania, co jest kluczowe w branży, gdzie każda minuta przestoju może skutkować znaczącymi stratami finansowymi i reputacyjnymi. Aplikacje służące do zamawiania jedzenia muszą być nie tylko funkcjonalne i intuicyjne dla użytkownika, ale także wydajne i elastyczne pod względem technologicznym.

Architektura mikroserwisowa jest jednym z najbardziej innowacyjnych podejść do tworzenia aplikacji webowych. W ostatnich latach zyskała ona na popularności, a jej zalety są doceniane przez coraz większą liczbę programistów. Wymaga ona jednak zupełnie innego podejścia niż architektura monolityczna, która jest najczęściej stosowana w aplikacjach tego typu. Różnice wynikają np. z innego podejścia do transakcyjności, synchronizacji danych oraz komunikacji między komponentami.

Zastosowanie architektury mikroserwisowej pozwala na elastyczne zarządzanie złożonymi systemami, umożliwiając jednocześnie łatwiejszą integrację z różnorodnymi usługami zewnętrznymi, takimi jak systemy płatności, zarządzanie dostawami czy integracja z mediami społecznościowymi. To podejście zapewnia również łatwiejsze skalowanie poszczególnych elementów systemu, co jest niezbędne w branży, gdzie wzrost popularności restauracji może gwałtownie zwiększyć zapotrzebowanie na zasoby informatyczne.

W pracy dyplomowej Autor prezentuje przykład aplikacji webowej zbudowanej zgodnie z najlepszymi praktykami architektury mikroserwisowej. Aplikacja ta będzie służyła do obsługi zamówień restauracyjnych z trzech perspektyw: klienta, restauracji oraz dostawcy.

Praca ta ma na celu nie tylko zaspokojenie potrzeb akademickich, ale także praktyczne pokazanie, jak nowoczesne technologie informatyczne mogą być wykorzystywane do rozwiązywania realnych problemów biznesowych. Opracowany system ma stanowić modelowy przykład aplikacji, która dzięki zastosowaniu architektury mikroserwisowej, może być łatwo dostosowywana do zmieniających się potrzeb rynku i rosnącej liczby użytkowników, co ma szczególne znaczenie w szybko rozwijającej się branży gastronomicznej.

\subsection{Cel pracy}

Celem niniejszej pracy było zaprojektowanie, stworzenie, wdrożenie oraz przetestowanie systemu informatycznego do obsługi zamówień restauracyjnych, opartego na architekturze mikroserwisowej. System ten ma na celu zapewnienie wydajnego, elastycznego i skalowalnego rozwiązania, które będzie mogło sprostać wymaganiom i oczekiwaniom trzech głównych grup użytkowników: zamawiających, restauracji oraz kurierów.

\subsection{Charakterystyka problemu}

Zastosowana architektura wymagała odpowiedniego podziału systemu informatycznego na mikroserwisy, które komunikują się między sobą w celu spełnienia wymagań biznesowych. System został podzielony na dwie części: kliencką i serwerową. Część kliencka jest aplikacją webową, uruchamianą w przeglądarce klienta, komunikującą się z częścią serwerową przy pomocy interfejsu REST. Część serwerowa została wykonana we wcześniej wspomnianej architekturze mikroserwisowej i zawiera całą logikę biznesową systemu. W tej pracy omówiono zastosowane techniki przy tworzeniu modułów oraz korzyści wynikające z wykorzystania tej specyficznej architektury. Ponadto, zaprezentowano krótkie zestawienie tej architektury z monolityczną, analizując ich dopasowanie do omawianego problemu.

\subsection{Przegląd podobnych rozwiązań}

W ramach tej pracy inżynierskiej Autor stwierdza, że nie był w stanie przeprowadzić dogłębnego porównania projektowanego systemu z istniejącymi na rynku rozwiązaniami takimi jak Uber Eats, Wolt, Pyszne.pl, Bolt Food. Powodem tego jest ograniczony dostęp do szczegółowych informacji na temat ich architektury serwerowej i rozwiązań technologicznych. Ponadto, porównanie funkcjonalności z perspektywy użytkownika końcowego nie było głównym celem pracy, gdyż nacisk położono nie na rozwój licznych funkcjonalności, lecz na badanie aspektów technicznych związanych z tworzeniem skalowalnych aplikacji webowych. W związku z tym analiza skupia się bardziej na kwestiach architektonicznych i wydajnościowych, a nie na bezpośredniej konkurencji z funkcjami oferowanymi przez wymienione platformy.


\subsection{Struktura pracy}

Kolejne rozdziały prezentują kolejne etapy prac nad prezentowanym systemem.

Na początku zostaną omówione popularne techniki tworzenia skalowalnych aplikacji webowych wykorzystane w niniejszej pracy inżynierskiej, w tym architektura mikroserwisowa. Zostanie następnie przedstawione jej porównanie do standardowej architektury monolitycznej.

Kolejny rozdział będzie poświęcony analizie wymagań budowanego systemu. Przedstawiony zostanie słownik dziedziny problemu, metoda wydzielania mikroserwisów, wymagania funkcjonalne i niefunkcjonalne.

W kolejnym rozdziale zostanie przedstawiona proponowana architektura systemowa i wdrożeniowa systemu, wraz z jej modelem C4. Opisane zostaną zastosowane mechanizmy persystencji danych, przechowywania i transportu komunikatów. Pokrótce opisane zostaną również mechanizmy uwierzytelniania i autoryzacji oraz zastosowane integracje z zewnętrznymi usługami.

Następny rozdział skupi się na implementacji systemu. Będzie zawierać szczegółowy opis procesu tworzenia obu segmentów systemu informatycznego, z uwzględnieniem wykorzystanych narzędzi, technik, ich zastosowania oraz przykładów ich użycia w systemie.

Przedostatni rozdział będzie poświęcony testowaniu systemu. Opisane zostaną różne rodzaje testów i sposób ich wykorzystania do sprawdzenia zgodności systemu z wymaganiami.

W końcowym rozdziale zostaną przedstawione wnioski wynikające z implementacji systemu i przeprowadzonych testów wydajnościowych. Na koniec przedstawione zostaną także perspektywy dalszego rozwoju systemu.