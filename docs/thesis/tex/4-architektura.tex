\clearpage % Rozdziały zaczynamy od nowej strony.

\section{Architektura systemu}

W tym rozdziale zostanie przedstawiona szczegółowa architektura projektowanego systemu. Celem tego rozdziału jest dostarczenie kompleksowego przeglądu wszystkich kluczowych komponentów, ich wzajemnych zależności, a także sposobu ich integracji.

\subsection{Podział na mikroserwisy}

Część serwerowa systemu została podzielona na mikroserwisy zgodnie z wcześniej wydzielonymi Ograniczonymi Kontekstami. Dodatkowo, poza serwisami związanych z Ograniczonymi Kontekstami, zostały wydzielone serwisy wspólne, które są wykorzystywane przez pozostałe elementy systemu. Są to:

\begin{itemize}

    \item \textbf{Dostawy} (ang. \textit{deliveries}) - implementacja kontekstu "Dostawa",
    \item \textbf{Zamówienia} (ang. \textit{orders}) - implementacja kontekstu "Zamówienie",
    \item \textbf{Płatności} (ang. \textit{payments}) - implementacja kontekstu "Płatność",
    \item \textbf{Restauracje} (ang. \textit{restaurants}) - implementacja kontekstu "Restauracja",
    \item \textbf{Zapytania} (ang. \textit{queries}) - serwis odpowiedzialny za budowanie bazodanowych projekcji danych oraz obsługę zapytań. Jest wykorzystywany przez wszystkie pozostałe serwisy oraz część kliencką systemu,
    \item \textbf{Sagi} (ang. \textit{sagas}) - serwis odpowiedzialny za obsługę/orkiestrację długo trwających procesór biznesowych. Implementuje wzorzec Saga. Jest wykorzystywany przez wszystkie pozostałe serwisy.

\end{itemize}

Wydzielono również moduł niebędący mikroserwisem, a jedynie biblioteką, która jest wykorzystywana przez wszystkie pozostałe serwisy. Jest to \textbf{Wspólny kod} (ang. \textit{shared}), który zawiera wspólne dla wszystkich serwisów klasy, interfejsy, konfiguracje, itp.

\subsection{Model C4 systemu}

W celu przedstawienia architektury systemu w sposób zrozumiały i przejrzysty, został wykorzystany model C4.

Model C4 (Context, Containers, Components, Code) to metoda wizualizacji architektury oprogramowania, zaprojektowana przez Simona Browna, która skupia się na dekompozycji systemu na różne poziomy abstrakcji. Model ten zaczyna od Kontekstu (poziom 1), pokazując zewnętrzne zależności systemu, przechodzi przez Kontenery (poziom 2), ukazujące główne części systemu (np. aplikacje webowe, bazy danych), aż do Komponentów (poziom 3), przedstawiających wewnętrzną budowę poszczególnych pojemników, kończąc na Kodzie (poziom 4), który pokazuje szczegóły implementacji. Model ten jest szeroko stosowany w projektowaniu i dokumentacji architektury oprogramowania, pozwalając na jasne i spójne przedstawienie skomplikowanych systemów.

TODO

\subsection{Protokoły i formaty komunikacji}

W celu zapewnienia komunikacji pomiędzy poszczególnymi komponentami systemu, zostały wykorzystane następujące protokoły i formaty komunikacji:

\begin{itemize}

    \item \textbf{Protokół HTTP} (ang. \textit{Hypertext Transfer Protocol}) - protokół komunikacyjny używany do przesyłania danych w Internecie, działający na zasadzie żądania i odpowiedzi między klientem a serwerem, charakteryzujący się bezstanowością i różnorodnymi metodami żądań, takimi jak GET czy POST. Bezpieczna wersja protokołu, HTTPS, dodatkowo zapewnia szyfrowanie danych, zwiększając prywatność i bezpieczeństwo transmisji. W projekcie został wykorzystany do komunikacji pomiędzy częścią kliencką systemu a serwisami serwerowymi.

    \item \textbf{Styl REST} (ang. \textit{Representational State Transfer}) - styl architektury oprogramowania używany do projektowania sieciowych systemów aplikacji, oparty na bezstanowych żądaniach i odpowiedziach, wykorzystujący standardowe metody HTTP i formaty wymiany danych, takie jak JSON czy XML, do tworzenia skalowalnych i elastycznych interfejsów API. REST akcentuje prostotę komunikacji sieciowej, umożliwiając łatwą integrację między różnymi systemami w Internecie.
    
    \item \textbf{Protokół gRPC} (ang. \textit{gRPC Remote Procedure Call}) - nowoczesny, otwartoźródłowy protokół zdalnego wywoływania procedur, opracowany przez Google, który umożliwia wydajną, szybką i bezpieczną komunikację między mikrousługami. Wykorzystuje format wymiany danych Protocol Buffers (Protobuf) i jest oparty na modelu klient-serwer, wspierając funkcje takie jak dwustronna komunikacja strumieniowa, kontrola przepływu i uwierzytelnienie oparta na certyfikatach TLS. W projekcie został wykorzystany do komunikacji pomiędzy serwisami a kolejką komunikatów oraz magazynem zdarzeń.

    \item \textbf{Format JSON} (ang. \textit{JavaScript Object Notation}) - lekki format wymiany danych, łatwy do czytania i pisania dla ludzi oraz prosty do parsowania i generowania dla maszyn, wykorzystujący tekstowe struktury do reprezentowania obiektowych danych, zazwyczaj składających się z par klucz-wartość i uporządkowanych list. Jest powszechnie stosowany w komunikacji internetowej, w szczególności w interfejsach API i konfiguracjach. W projekcie został wykorzystany do przesyłania danych pomiędzy częścią kliencką systemu a serwisami.
    
    \item \textbf{Format XML} (ang. \textit{Extensible Markup Language}) - elastyczny i rozszerzalny format danych oparty na znacznikach, zaprojektowany do przechowywania i transportu danych w sposób zarówno czytelny dla maszyn, jak i ludzi, często używany w różnorodnych aplikacjach internetowych i konfiguracjach systemowych. Charakteryzuje się ścisłą strukturą, z hierarchicznie uporządkowanymi elementami i atrybutami, co sprawia, że jest idealny do reprezentowania złożonych struktur danych. W projekcie został wykorzystany do serializacji i deserializacji wiadomości: zdarzeń, komend i zapytań.

\end{itemize}

\subsection{Persystencja danych}

TODO magazyn zdarzeń

TODO projekcyjna baza danych

\subsection{Kolejka komunikatów i magazyn zdarzeń}

TODO axon server

\subsection{Uwierzytelnianie i autoryzacja}

TODO Auth0

TODO OAuth2 (Authorization code grant)

TODO JWT

\subsection{Integracje zewnętrzne}

TODO Google Maps Platform

TODO Stripe

TODO Mailjet

TODO invoice generator