\clearpage % Rozdziały zaczynamy od nowej strony.

\section{Podsumowanie}

\subsection{Wnioski}

Celem pracy dyplomowej było zaprojektowanie i implementacja mikroserwisowego systemu informatycznego do zarządzania zamówieniami restauracyjnymi. System został stworzony zgodnie ze zdefiniowanymi założeniami funkcjonalnymi i niefunkcjonalnymi, przy użyciu technologii dobranych na podstawie analizy porównawczej. Zastosowane techniki i narzędzia umożliwiły zapewnienie wymaganej wydajności oraz skalowalności systemu, co potwierdziły przeprowadzone testy wydajnościowe.

W ramach pracy dyplomowej zaimplementowano wszystkie funkcjonalności określone w wymaganiach funkcjonalnych. Dodatkowo udało się zrealizować część funkcjonalności dodatkowych, które nie były pierwotnie zdefiniowane, takie jak np. możliwość śledzenia lokalizacji kuriera w czasie rzeczywistym.

Zrealizowano również wszystkie wymagania niefunkcjonalne. Skalowalność osiągnięto dzięki architekturze mikroserwisowej, kolejkom komunikatów i odpowiedniej infrastrukturze sieciowej. Wydajność systemu, zmierzona podczas testów, spełnia założenia przetwarzania minimum 50 żądań na sekundę. System jest monitorowalny dzięki narzędziom takim jak Prometheus, Grafana, Loki, a także testowalny, co udowodniono przez testy jednostkowe, integracyjne i wydajnościowe. Wymagania dotyczące bezpieczeństwa zostały spełnione przez zastosowanie autoryzacji JWT i platformy Auth0.

\subsection{Możliwości dalszego rozwoju}

Ze względu na zastosowanie architektury mikroserwisowej, zorientowanej na zdarzenia, system łatwo można rozbudować, dodając nowe funkcjonalności. W ramach dalszego rozwoju można skupić się na jednym z następujących obszarów:

\textbf{Rozszerzenie funkcjonalności aplikacji} - Rozwój aplikacji może obejmować dodanie nowych funkcji, np. analizę i wizualizację danych związanych ze sprzedażą i wydajnością kurierów.

\textbf{Podniesienie wydajności systemu i dalsze testy wydajnościowe} - Możliwe jest dalsze zwiększenie wydajności systemu, np. poprzez zastosowanie innych narzędzi lub modyfikacje w modelowaniu systemu. Warto również przeprowadzić testy wydajnościowe z większą liczbą użytkowników.

\textbf{Ulepszenie warstwy prezentacji części klienckiej} - Kolejnym obszarem rozwoju może być ulepszenie interfejsu użytkownika, np. poprzez rozwój aplikacji mobilnych na systemy Android i iOS.