\clearpage % Rozdziały zaczynamy od nowej strony.
\section{Wybór tematu pracy dyplomowej}

Wybór tematu pracy dyplomowej został podyktowany chęcią pogłębienia wiedzy oraz zaprezentowania technik tworzenia skalowalnych systemów mikroserwisowych na podstawie konkretnej aplikacji.

Wybraną przez Autora dziedziną do tego celu jest branża gastronomiczna. W ostatnich latach, wraz z rozwojem techniki, zaczęło powstawać wiele aplikacji służących do zamawiania jedzenia z restauracji. Aplikacje te cieszą się ogromną popularnością, więc ich twórcy muszą zapewnić odpowiednią skalowalność systemu, aby móc obsłużyć duży ruch. Jednym z rozwiązań tego problemu jest zastosowanie skalowalnych architektur, np. architektury mikroserwisowej.

Architektura mikroserwisowa jest jednym z najbardziej innowacyjnych podejść do tworzenia aplikacji webowych. W ostatnich latach zyskała ona na popularności, a jej zalety są doceniane przez coraz większą liczbę programistów. Wymaga ona jednak zupełnie innego podejścia niż architektura monolityczna, która jest najczęściej stosowana w aplikacjach tego typu. Różnice wynikają np. z innego podejścia do transakcyjności, synchronizacji danych oraz komunikacji między komponentami.

W pracy dyplomowej Autor prezentuje przykład aplikacji webowej zbudowanej zgodnie z najlepszymi praktykami architektury mikroserwisowej. Aplikacja ta będzie służyła do obsługi zamówień restauracyjnych z trzech perspektyw: klienta, restauracji oraz dostawcy.