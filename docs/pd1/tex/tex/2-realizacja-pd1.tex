\clearpage % Rozdziały zaczynamy od nowej strony.
\section{Realizacja Pracowni}

% TODO pewnie to przeniosę
\subsection{Harmonogram prac}

Na początku semestru został sporządzony harmonogram prac, który został przedstawiony w tabeli \ref{table:harmonogram}.

\begin{longtable}{| m{0.67\linewidth} | r |}
    \caption{Harmonogram prac.}
    \label{table:harmonogram} \\

    \hline
    % Nagłówek tabeli wyrównujemy do środka
    \multicolumn{1}{c|}{Kamień milowy} & \multicolumn{1}{c|}{Przewidywana data ukończenia} \\ \hline\hline \endfirsthead \endfoot
    \hline \endlastfoot

    Zamodelowanie zdarzeń biznesowych przy pomocy techniki Event Storming & 27.03.2023 \\ \hline
    Wstępny wybór technologii & 17.04.2023 \\ \hline
    Przygotowanie środowiska programistycznego & 24.04.2023 \\ \hline
    Weryfikacja wykorzystania wybranych technologii & 8.05.2023 \\ \hline
    Ustalenie wymagań funkcjonalnych i niefunkcjonalnych & 22.05.2023 \\ \hline
    Wykonanie makiet interfejsu użytkownika & 5.06.2023 \\ \hline
    Ustalenie formatów i technik wymiany danych & 12.06.2023 \\ \hline
    Przygotowanie mechanizmów uwierzytelniania i autoryzacji użytkowników & 19.06.2023 \\ \hline
    Zaprojektowanie architektury i infrastruktury aplikacji & 26.06.2023 \\ \hline
    Główna implementacja & 2.10.2023 \\ \hline
    Wdrożenie produkcyjne aplikacji & 9.10.2023 \\ \hline
    Przygotowanie i wykonanie automatycznych testów akceptacyjnych & 23.10.2023 \\ \hline
    Przygotowanie i wykonanie automatycznych testów wydajnościowych & 6.11.2023\\ \hline
    Przygotowanie raportu z wydajności i skalowalności systemu & 13.11.2023 \\ \hline
    Przygotowanie wstępnego tekstu pracy inżynierskiej & 18.12.2023 \\ \hline
    Przygotowanie finalnego tekstu pracy inżynierskiej & 8.01.2024 \\ \hline
    Złożenie pracy inżynierskiej & 15.01.2024 \\ \hline
\end{longtable}

Zakres Pracowni Dyplomowej 1 zakończył się wraz z ukończeniem kamienia milowego \textit{Wykonanie makiet interfejsu użytkownika}.

% Podrozdział pierwszego poziomu
\subsection{Zamodelowanie zdarzeń biznesowych}

W celu zrozumienia procesów oraz zidentyfikowania potencjalnych mikroserwisów w tworzonej aplikacji, przeprowadzono sesję Event Stormingu. Metoda ta umożliwia wizualizację i modelowanie procesów biznesowych w ramach systemu, co jest kluczowe przy tworzeniu złożonych systemów informatycznych. W niniejszym rozdziale przedstawione zostaną wyniki sesji Event Stormingu, która została przeprowadzona z wykorzystaniem narzędzia Miro.

% Tabela wielostronicowa, 4 kolumny
% Kolumny typu m{} oznaczają kolumny o stałej szerokości z zawijaniem wierszy
% Wyrównywane są domyślnie do lewej; aby ustawić inne wyrównanie,
% stosujemy \multicolumn{1} tak jak poniżej
\begin{longtable}{| c | m{0.58\linewidth} | r | m{0.1\linewidth} |}
    \caption{Tabela wielostronicowa.}
    \label{table:koszty} \\

    \hline
    % Nagłówek tabeli wyrównujemy do środka
    Lp & \multicolumn{1}{c|}{Treść} & \multicolumn{1}{c|}{Kwota} & \multicolumn{1}{m{0.1\linewidth}|}{Wariant opłaty} \\ \hline\hline \endfirsthead \endfoot
    \hline \endlastfoot

    1  & Lorem ipsum dolor sit amet, consectetur adipiscing elit, sed do eiusmod tempor incididunt ut labore et dolore magna aliqua. & 111 111,11 zł & \multicolumn{1}{c|}{WAR1} \\ \hline
    2  & Lorem ipsum dolor sit amet, consectetur adipiscing elit, sed do eiusmod tempor incididunt ut labore et dolore magna aliqua. & 22 222,22 zł & \multicolumn{1}{c|}{WAR1} \\ \hline
    3  & Lorem ipsum dolor sit amet, consectetur adipiscing elit, sed do eiusmod tempor incididunt ut labore et dolore magna aliqua. & 33 333,33 zł & \multicolumn{1}{c|}{WAR1} \\ \hline
    4  & Lorem ipsum dolor sit amet, consectetur adipiscing elit, sed do eiusmod tempor incididunt ut labore et dolore magna aliqua. & 444 444,44 zł & \multicolumn{1}{c|}{WAR1} \\ \hline
    5  & Lorem ipsum dolor sit amet, consectetur adipiscing elit, sed do eiusmod tempor incididunt ut labore et dolore magna aliqua. & 55 555,55 zł & \multicolumn{1}{c|}{WAR1} \\ \hline
    6  & Lorem ipsum dolor sit amet, consectetur adipiscing elit, sed do eiusmod tempor incididunt ut labore et dolore magna aliqua. & 66 666,66 zł & \multicolumn{1}{c|}{WAR1} \\ \hline
    7  & Lorem ipsum dolor sit amet, consectetur adipiscing elit, sed do eiusmod tempor incididunt ut labore et dolore magna aliqua. & 777 777,77 zł & \multicolumn{1}{c|}{WAR1} \\ \hline
    8  & Lorem ipsum dolor sit amet, consectetur adipiscing elit, sed do eiusmod tempor incididunt ut labore et dolore magna aliqua. & 8 888,88 zł & \multicolumn{1}{c|}{WAR1} \\ \hline
    9  & Lorem ipsum dolor sit amet, consectetur adipiscing elit, sed do eiusmod tempor incididunt ut labore et dolore magna aliqua. & 999 999,99 zł & \multicolumn{1}{c|}{WAR1} \\ \hline
    10 & Lorem ipsum dolor sit amet, consectetur adipiscing elit, sed do eiusmod tempor incididunt ut labore et dolore magna aliqua. & 111 111,11 zł & \multicolumn{1}{c|}{WAR2} \\ \hline
    11 & Lorem ipsum dolor sit amet, consectetur adipiscing elit, sed do eiusmod tempor incididunt ut labore et dolore magna aliqua. & 22 222,22 zł & \multicolumn{1}{c|}{WAR2} \\ \hline
    12 & Lorem ipsum dolor sit amet, consectetur adipiscing elit, sed do eiusmod tempor incididunt ut labore et dolore magna aliqua. & 33 333,33 zł & \multicolumn{1}{c|}{WAR2} \\ \hline
    13 & Lorem ipsum dolor sit amet, consectetur adipiscing elit, sed do eiusmod tempor incididunt ut labore et dolore magna aliqua. & 444 444,44 zł & \multicolumn{1}{c|}{WAR2} \\ \hline
    14 & Lorem ipsum dolor sit amet, consectetur adipiscing elit, sed do eiusmod tempor incididunt ut labore et dolore magna aliqua. & 55 555,55 zł & \multicolumn{1}{c|}{WAR2} \\ \hline
    15 & Lorem ipsum dolor sit amet, consectetur adipiscing elit, sed do eiusmod tempor incididunt ut labore et dolore magna aliqua. & 66 666,66 zł & \multicolumn{1}{c|}{WAR2} \\ \hline
       & \multicolumn{1}{r|}{\textbf{Suma:}} & \textbf{7 777 777,77 zł} &
\end{longtable}

\kant[2]

% Nagłówki kolejnych poziomów, dla zapełnienia spisu treści
\subsection{Wstępny wybór technologii}
\subsubsection{Część serwerowa}
\kant[2]
\subsubsection{Część kliencka}
\kant[3]

\subsection{Przygotowanie środowiska programistycznego}
\kant[9] Lorem ipsum dolor sit amet, consectetur adipiscing elit \cite{benzmuller2014}, \cite{goedel95}, \cite{wang97}, \cite{koons2005}.
\subsubsection{Środowisko lokalne}
\kant[2]
\subsubsection{Środowisko wdrożeniowe}
\kant[3]


\subsection{Weryfikacja wykorzystania wybranych technologii}
\kant[9] Lorem ipsum dolor sit amet, consectetur adipiscing elit \cite{benzmuller2014}, \cite{goedel95}, \cite{wang97}, \cite{koons2005}.

\subsection{Ustalenie wymagań funkcjonalnych i niefunkcjonalnych}
\kant[9] Lorem ipsum dolor sit amet, consectetur adipiscing elit \cite{benzmuller2014}, \cite{goedel95}, \cite{wang97}, \cite{koons2005}.
\subsubsection{Wymagania funkcjonalne}
\kant[2]
\subsubsection{Wymagania niefunkcjonalne}
\kant[3]

\subsection{Wykonanie makiet interfejsu użytkownika}
\kant[9] Lorem ipsum dolor sit amet, consectetur adipiscing elit \cite{benzmuller2014}, \cite{goedel95}, \cite{wang97}, \cite{koons2005}.

% Twierdzenia i dowody
% Założenie
\begin{assumption} \label{ass:1}
    $ [\![ \ \phi \ ]\!] \Longrightarrow [\![ \ P(\phi); \neg P(\phi) \ ]\!]$
\end{assumption}
% Aksjomat
\begin{axiom}[Dualność] \label{axiom:1}
    $\neg P(\phi) \Leftrightarrow P(\neg \phi)$, równoważnie $P(\phi) \Leftrightarrow \neg P(\neg \phi)$
\end{axiom}
\begin{axiom}[Całkowitość] \label{axiom:2}
    $ \left( P(\phi) \wedge \forall x: \phi(x) \Rightarrow \psi(x) \right) \Rightarrow P(\psi) $
\end{axiom}
\begin{axiom}[Absolutność] \label{axiom:3}
    $ P(\phi) \Rightarrow \Box P(\phi) $
\end{axiom}
% Definicja
\begin{definition} \label{def:1}
    $ G(x) \Leftrightarrow \forall \phi: \left( P(\phi) \Rightarrow \phi(x) \right) $
\end{definition}
\begin{definition} \label{def:2}
    $ \phi \ ess \ x \Leftrightarrow \phi(x) \wedge \forall \psi \left( \psi(x) \Rightarrow \Box \forall y \left( \phi(y) \Rightarrow \psi(y) \right) \right)  $
\end{definition}
\begin{axiom} \label{axiom:4}
    P(G)
\end{axiom}
% Lemat
\begin{lemma} \label{lemma:1}
    $ P(\phi) \Rightarrow \Diamond \exists x : \phi(x) $
\end{lemma}
\begin{proof}
    Dowód pomijamy, bo jest trywialny :)
\end{proof}
\begin{lemma} \label{lemma:2}
    $ \Diamond \exists x : G(x) $
\end{lemma}
\begin{proof}
    Natychmiastowy wniosek z aksjomatu \ref{axiom:4} i lematu \ref{lemma:1}.
\end{proof}
\begin{lemma} \label{lemma:3}
    $ G(x) \Rightarrow G \ ess \ x $
\end{lemma}
\begin{proof}
    Poprzez podstawienie do definicji \ref{def:2}.
\end{proof}
\begin{definition} \label{def:3}
    $ E(x) \Leftrightarrow \forall \phi \left( \phi \ ess \ x \Rightarrow \Box\ \exists x: \phi(x) \right) $
\end{definition}
\begin{axiom} \label{axiom:5}
    P(E)
\end{axiom}
% Twierdzenie
\begin{theorem}
    $ \Box\ \exists x : G(x) $
\end{theorem}
\begin{proof}
    Na podstawie definicji \ref{def:1}, lematu \ref{lemma:3} i aksjomatu \ref{axiom:5}.
\end{proof}
