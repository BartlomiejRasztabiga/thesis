\clearpage % Rozdziały zaczynamy od nowej strony.
\section{Wybór tematu pracy dyplomowej}

Wybór tematu pracy dyplomowej został podytkowany chęcia pogłębienia wiedzy oraz zaprezentowaniu technik tworzenia skalowalnych systemów mikroserwisowych.

Odpowiednią dziedziną do tego celu jest branża gastronomiczna. W ostatnich latach, wraz z rozwojem technologii, zaczęło powstawać wiele aplikacji służących do zamawiania jedzenia z restauracji. Aplikacje te cieszą się ogromną popularnością, więc ich twórcy muszą zapewnić odpowiednią skalowalność systemu, aby móc obsłużyć duże obciążenia. Jednym z rozwiązań tego problemu jest zastosowanie skalowalnych architektur, np. architektury mikroserwisowej.

Architektura mikroserwisowa jest jednym z najbardziej innowacyjnych podejść do tworzenia aplikacji webowych. W ostatnich latach zyskała ona na popularności, a jej zalety są doceniane przez coraz większą liczbę programistów. Wymaga ona jednak zupełnie innego podejścia niż architektura monolityczna, która jest najczęściej stosowana w aplikacjach tego typu. Różnice wynikają np. z innego podejścia do transakcyjności, synchronizacji danych oraz komunikacji między komponentami.

W pracy dyplomowej Autor zaprezentuje przykład aplikacji webowej zbudowanej zgodnie z najlepszymi praktykami architektury mikroserwisowej. Aplikacja ta będzie służyła do obsługi zamówień restauracyjnych z trzech perspektyw: klienta, restauracji oraz dostawcy.

% Przykładowa tabela: wyśrodkowana i renderowana
% w miejscu wstawienia: !h = !h[ere]
% Domyślnie tabele trafiają na górę strony
\begin{table}[!h] \centering
    % Podpis tabeli umieszczamy od góry
    \caption{Przykładowa tabela.}
    \label{tab:tabela1}

    % Tabela z trzema kolumnami:
    % dwie wyrównanie do środka [c], a ostatnia do prawej [r]
    % szerokość kolumn automatyczna (równa szerokości tekstu)
    \begin{tabular}{| c | c | r |} \hline
        Kolumna 1       & Kolumna 2 & Liczba \\ \hline\hline
        cell1           & cell2     & 60     \\ \hline
        cell4           & cell5     & 43     \\ \hline
        cell7           & cell8     & 20,45  \\ \hline
        % Komórka o szerokości dwóch kolumn, wyrównana do prawej
        % Przypisy dolne w tabelach wstawiamy przez \tablefootnote
        \multicolumn{2}{|r|}{Suma\tablefootnote{Table footnote.}} & 123,45 \\ \hline
    \end{tabular}

\end{table}

Lorem ipsum dolor sit amet.
